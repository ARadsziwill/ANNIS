\hyphenation{ANNIS IFrame}

\epigraph{Nicht weil es schwer ist, wagen wir es nicht; weil wir es nicht wagen, ist es schwer.}
         {Seneca}

\vspace{310pt}
\begin{minipage}[b!]{1.0\linewidth}
\section*{Danksagung}
Diese Diplomarbeit markiert das lang ersehnte Ende meiner Studienzeit. Ich hatte das Glück, zum Abschluss an einem Produkt zu arbeiten, mit dem die Linguistik den Weg zur Erkenntnis vielleicht etwas bequemer beschreiten kann. Mein besonderer Dank gilt Anke, Amir und Hagen für deren engagierte Unterstützung und die großartige Zusammenarbeit. Die Arbeit mit Euch hat mir sehr viel gegeben und mein Denken verändert. Dem Stef danke ich für seine unermüdliche Fehlersuche und die Hilfe bei der Gestaltung der fertigen Exemplare. Ich danke Philipp dafür, dass ich so frei Arbeiten konnte.
\end{minipage}

\newpage
\section{Einleitung}

Die vorliegende Arbeit stellt die Entwicklung einer Benutzerschnittstelle für die Suche
in Mehrebenen-Korpora vor. Korpora werden in der Linguistik als digitale, kontrolliert erhobene Textsammlungen definiert, die der Sprachwissenschaft als empirische Forschungsgrundlage dienen und zu diesem Zweck linguistisch interpretiert -- annotiert -- sind. Geschieht diese Interpretation der Sprachdaten auf unterschiedlichen und separaten Analyseebenen, so ist entsprechend von Mehrebenen-Korpora die Rede.

Die Suche in solchen annotierten Textdaten ist in der modernen Sprachwissenschaft bei der Bearbeitung vieler Forschungsfragen ein unentbehrliches Hilfsmittel geworden, von dem mehr und mehr Linguisten Gebrauch machen. Bei den durchsuchbaren Sprachdaten kann es sich je nach Forschungsfrage um unterschiedlichste geschriebene Texte oder um transkribierte (verschriftlichte) gesprochene Sprache handeln. Der Nutzen bei der systematischen, computergestützten Suche in diesen Korpusdaten liegt oftmals einfach in der Belegsuche, vor allem aber bieten statistische Analysen über annotierten Textdaten eine nachvollziehbare und valide Grundlage für die Etablierung, Erweiterung, Bestätigung oder Widerlegung linguistischer Hypothesen oder Theorien. 
Leider wird allgemein die Nutzerfreundlichkeit und Effizienz der existierenden Suchwerkzeuge für die Korpussuche bemängelt, was bei dem geringen Alter dieses technischen Wissenschaftszweigs und der Komplexität der Anforderungen nicht verwunderlich ist. 

In diesem Sinne ist es das Ziel der vorliegenden Diplomarbeit, eine innovative, web-basierte und nutzerfreundliche Benutzerschnittstelle für Mehrebenen-Korpora zu entwickeln. Das Besondere an diesem Interface ist eine flexible, leicht verständliche Darstellung der Korpusdaten und Suchergebnisse sowie ein Werkzeug zur grafischen Formulierung von Anfragen.
Diese Arbeit ist somit strikt lösungsorientiert und liefert zu dem vorliegenden theoretischen Teil eine praktisch nutzbare Umsetzung. Dazu gehört unter anderem die umfassende Implementierung der Benutzerschnittstelle und deren Übergabe in den Produktivbetrieb der Korpuslinguistik am Institut für deutsche Sprache und Linguistik der Humboldt-Universität zu Berlin.

Auf der einen Seite spielt bei der Erstellung der Schnittstelle die Realisierung innovativer Funktionalitäten eine wichtige Rolle, auf der anderen Seite wird gesteigerter Wert auf Nutzerfreundlichkeit gelegt, denn es ist allgemein bekannt, dass diese ein wichtiges Qualitätskriterium für Hard- und Software ist (Mayhew und Bias, 1994). Um eine nachhaltig gebrauchstaugliche Anwendung zu entwickeln, muss deshalb ingenieurspsychologischen Aspekten besondere Beachtung zukommen. Aus diesem Grund werden bei der Erstellung der Benutzerschnittstelle unter anderem ingenieurspsychologische Gestaltungsprinzipien sowie eingangs erhobene Umfragedaten berücksichtigt.

%Insbesondere die Anwendung ingenieurspsychologischer Gestaltungsprinzipien sowie die Verwendung von erhobenen Umfragedaten ermöglichen den Entwurf und die Umsetzung einer neuartigen flexiblen linguistischen Suche. 

%Die Suche in annotierten Textdaten ist ein wichtiges Hilfsmittel in der modernen Sprachwissenschaft, von dem mehr und mehr Linguisten Gebrauch machen. Dabei kann es sich um verschiedenartige geschriebene Texte oder um transkribierte gesprochene Sprache handeln. Die Vorteile bei der systematischen, computergestützten Suche in diesen Korpusdaten liegen vor allem in der Belegsuche und statistischen Analysen über annotierten Textdaten \citep[][]{wynneXXXXsearching}.

Der hier vorgestellte Beitrag gliedert sich ein in die Arbeit des Sonderforschungsbereiches SFB 632 ''Informationsstruktur\footnote{Unter Informationsstruktur versteht der SFB 632 die Strukturierung von sprachlicher Information, typischerweise zum Zweck der Optimierung des Informationstransfers im Diskurs.}: Die sprachlichen Mittel der Gliederung von Äußerung, Satz und Text''\footnote{http://www.sfb632.uni-potsdam.de/main3322.html}. Hier soll im Teilprojekt D1 ''Linguistische Datenbank für Informationsstruktur: Annotation und Retrieval''\footnote{http://www.sfb632.uni-potsdam.de/\~{ }d1/} eine flexible Suchplattform für Mehrebenen-Korpora geschaffen und einer breiten Nutzergruppe zur Verfügung gestellt werden. Dies ist nicht nur konzeptionell-theoretisch, sondern auch technisch gesehen eine große Herausforderung, sowohl bezogen auf die Entwicklung eines Suchsystems und einer Anfragesprache als auch auf die Erarbeitung einer allgemeinen Benutzerschnittstelle.

Im SFB 632 wurde bereits die Anfragesprache \emph{ANNIS Query Language} \emph{AQL} (siehe Abschnitt \ref{sec:AQL}) entwickelt. Darauf aufbauend, dient \emph{ANNIS} \citep[]{annis}\footnote{ANNIS Homepage: http://www.sfb632.uni-potsdam.de/\~{ }d1/annis/} als web-basierte grafische Benutzeroberfläche. \emph{AQL} ist syntaktisch an die Baumstruktur linguistischer Daten angelehnt. Sie beschreibt die gesuchten Teilgraphen durch die Eigenschaften von Knoten und Kanten. Deshalb bietet \emph{AQL} Vorteile in der Vermittlung für unerfahrene Korpusnutzer  und ist für diverse Mehrebenen-Korpora gut geeignet. In \emph{ANNIS 1.0} werden die Daten im flexiblen XML-stand-off-Format \emph{PAULA} (\citet[]{dipper2005paula}, \citet[]{woerner2006}) gespeichert und für die Suche in eine geeignete Hauptspeicherrepräsentation des Graphen überführt. Die stärkste Einschränkung dieser Vorgehensweise ist die Platzbegrenzung des Hauptspeichers. Insbesondere kann der Java Heap\footnote{\emph{Java} Ausführungsumgebungen organisieren deren virtuellen Speicher in einer Heap-Struktur, die einen zusammenhängenden Block im Hauptspeicher benötigen.} auf 32bit-Systemen eine Gesamtgröße von 1.996.800kb ($\approx$2GB) nicht überschreiten. Damit ist diese Herangehensweise für große Korpora ungeeignet.

Deswegen soll zukünftig die von \citet{dddquery-springer} vorgestellte Anfragesprache zur Suche verwendet werden. \emph{DDDQuery} beinhaltet ein Datenbankmodell zur Ablage von Korpora und wird mit Hilfe eines Compilers in SQL-Anfragen übersetzt. \emph{DDDQuery}  ist eine sehr umfangreiche bzw. komplexe Anfragesprache, mit der die meisten Endbenutzer nicht konfrontiert werden können. Daher soll \emph{AQL} zukünftig nach \emph{DDDQuery} überführt werden.

%Ziel der vorliegenden Diplomarbeit ist es, eine neue web-basierte Benutzerschnittstelle für \emph{AQL} über Mehrebenen-Korpora zu entwickeln. Besonderheiten dieses Interfaces sind die flexible, verständliche Darstellung der Korpusdaten und Suchergebnisse sowie ein Werkzeug zur grafischen Erstellung von Anfragen. 
%
%Diese Arbeit ist lösungsorientiert und bietet eine praktisch nutzbare Umsetzung an. Dazu gehört auch die umfassende Implementierung der Benutzerschnittstelle und deren Übergabe in den Produktivbetrieb der \emph{Korpuslinguistik} am \emph{Institut für deutsche Sprache und Linguistik} der \emph{Humboldt-Universität zu Berlin}. 
%
%Das Hauptaugenmerk soll aber nicht ausschließlich auf den Funktionalitäten  der Anwendung liegen. Es ist weitgehend bekannt, dass Nutzerfreundlichkeit ein wichtiges Qualitätskriterium von Hard- und Software ist \citep[][]{mayhew94}. Um eine langfristig nachhaltige Anwendung zu entwickeln, muss deshalb ingenieurspsychologischen Aspekten besondere Beachtung zukommen. 

\subsection{Verwandte Arbeiten}\label{sec:Verwandte Arbeiten}

Korpusdaten sind durch ihre Struktur und Größe nicht immer einfach zu handhaben. Deswegen werden seit Jahren Suchsysteme für spezielle Daten entwickelt. Im englischsprachigen Raum sind die bekanntesten Systeme die \emph{Penn Treebank} mit dem Suchinterface \emph{CorpusSearch} \citep[]{marcus1994building}\footnote{Treebank Homepage: http://www.cis.upenn.edu/\~{}treebank/} und das \emph{British National Corpus} mit dem Suchinterface \emph{Xaira} \citep[]{burnard2000reference}\footnote{BNC Homepage: http://www.natcorp.ox.ac.uk/}, für den deutschsprachigen Raum ist es das \emph{TIGER-}Korpus mit \emph{TIGERSearch} \citep[]{lezius-diss}\footnote{\emph{TIGERSearch} Homepage: http://www.ims.uni-stuttgart.de/projekte/TIGER/TIGERSearch/}. \emph{Xaira} und \emph{TIGERSearch} mit dem Zusatzwerkzeug \emph{TIGERRegistry} können darüber hinaus vielfältige andere XML-Korpusdaten durchsuchen. Leider besitzt fast jedes Korpus eine eigene Suchplattform/Benutzerschnittstelle mit individuell sehr unterschiedlichen Ansätzen. Die Suche über mehrere verschiedenartige Korpora ist daher auch dann nicht möglich, wenn sie vergleichbare Annotationen (Annotationsebenen und -kategorien) besitzen. 

\cite{futterleib2007} beschäftigt sich in ihrer Magisterarbeit mit den allgemeinen Aspekten der Visualisierung von reich annotierten linguistischen Daten. Sie spezifiziert Anforderungen für die Darstellung linguistischer Daten basierend auf Interviews mit Vertretern verschiedener SFB-Projekte. Wenngleich sie Implementierungsmöglichkeiten vorstellt, liegt der Schwerpunkt jedoch in der Analyse einzelner Annotationsebenen sowie allgemeiner Aspekte der Visualisierung. Darüber hinaus werden fortgeschrittene Techniken der Web-Programmierung und Systemarchitektur nicht betrachtet.

\subsection{Aufbau dieser Arbeit}

Diese Arbeit ist in fünf Hauptteile eingeteilt. Zunächst werden in Abschnitt \ref{sec:Korpussuche} die grundlegenden Voraussetzungen für die Korpussuche mit \emph{ANNIS 2.0} vorgestellt. Diese sind das Datenmodell (Abschnitt \ref{sec:Korpussuche.Datenmodell}), dessen Repräsentation über spezielle Datenformate (Abschnitt \ref{sec:Korpussuche.Datenformat}) sowie die Anfragesprachen (Abschnitt \ref{sec:Korpussuche.Anfragesprache}) und die Benutzerschnittstelle (Abschnitt \ref{sec:Korpussuche.Benutzerschnittstelle}).

Danach werden in Abschnitt \ref{sec:Usability} theoretische Grundüberlegungen zur Erstellung benutzerfreundliche Software vorgestellt. Diese werden dann in Abschnitt \ref{sec:Nutzerbefragung} durch die Ergebnisse der durchgeführten Benutzerbefragung konkretisiert. Dazu werden in Abschnitt \ref{sec:Personas} geeignete Personas als anschauliches Entwurfswerkzeug vorgestellt. Abschließend werden mit der Betrachtung der Vorteile von Desktop-Anwendungen gegenüber traditionellen Web-Anwendungen die Grundlagen für die Entwicklung einer modernen web-basierten Lösung, die den Benutzern die Vertrautheit von Desktop-Anwendungen gibt, erörtert. 

Die konkrete Umsetzung dieser modernen Web-Anwendung wurde im Rahmen dieser Arbeit durchgeführt und wird in Abschnitt \ref{sec:ANNIS 2.0} vorgestellt. Besonderes Augenmerk liegt an dieser Stelle auf der Implementierung einer fensterorientierten Benutzerschnittstelle (Abschnitt \ref{sec:ExtJS}) und den dafür notwendigen Technologien. Weiterhin werden die Lösungen verschiedener Darstellungen von Ergebnissen erläutert und mit dem grafischen Anfrageneditor \emph{ANNIS.build} (siehe Abschnitt \ref{sec:ANNIS.build}) wird ein benutzerfreundliches Werkzeug zur einfachen Formulierung von Suchanfragen vorgestellt. Danach finden der Authentifizierungs- und Autorisierungsmechanismus sowie das Zitierungssystem für Suchanfragen Erwähnung. 

Abschließend wird in Abschnitt \ref{sec:Diskussion} das Fazit dieser Arbeit gezogen und ein Ausblick auf mögliche Erweiterungen des Systems besprochen. Damit ist die Grundlage für einen schrittweisen Ausbau der abgelieferten Benutzerschnittstelle gegeben.

Dieser Arbeit ist eine CD-ROM beigelegt. Sie enthält neben der digitalen Version dieses Dokuments die Quellen des Benutzerinterfaces \emph{ANNIS 2.0} (Verzeichnis /AnnisWEB). Diese sind in ein Eclipse\footnote{Eclipse Homepage: http://eclipse.org}-Projekt eingebunden, können aber ebenso mit anderen Werkzeugen bearbeitet, übersetzt und paketiert (siehe /AnnisWEB/build.xml) werden.

\newpage
\section{Einführung in die Korpussuche}\label{sec:Korpussuche}

Die Korpussuche ist ein objektives Werkzeug zur Überprüfung von linguistischen Hypothesen bezogen auf eine bestimmte aufbereitete Datenbasis. Sie dient der Belegsuche und kann damit die Grundlage für quantitative Auswertungen liefern. Die Suche in linguistisch annotierten Daten ist für objektive Resultate unverzichtbar und ergänzt immer mehr introspektive Methoden, die innere Erfahrungen rekurrieren. Eine anschauliche Diskussion dieser zwei Traditionen findet sich in \cite{fillmore1992armchair}.

\cite{wynneXXXXsearching} beschreibt das Suchen nach Wörtern, Phrasen und Mustern als grundlegendes Anwendungsgebiet der Korpussuche. Daraus leiten sich dann die erweiterte Suchtechnik des Untersuchens von Konkordanzen\footnote{Bei der Untersuchung von Konkordanzen werden Fundstellen im Zusammenhang zu den sie umgebenden Worten betrachtet.} sowie der Bedarf nach Sortierung, Kategorisierung und Verfeinerung von Suchergebnissen ab.

Die Umsetzung einer Korpussuche umfasst immer die drei Teilaspekte \emph{Datenmodell}, \emph{Datenformat} und \emph{Anfragesprache}. Alle drei haben gleichermaßen Einfluss darauf, welche linguistischen Fragestellungen sich mit dem daraus entstehenden Gesamtsystem beantworten lassen. Korpussysteme werden meist unter Berücksichtigung einer konkreten Forschungsfrage erstellt.
Deshalb eignen diese sich mitunter nur für einen begrenzte Klasse von verwandten Untersuchungen. Dennoch sind die möglichen Fragestellungen und Ziele, die der Korpussuche zugrunde liegen, vielfältig. Um die Auswirkungen der einzelnen Teilaspekte im Verlauf dieser Arbeit besser beurteilen zu können, werden exemplarische Fragestellungen nachfolgend separat vorgestellt.

\newpage
\subsection{Datenmodell}\label{sec:Korpussuche.Datenmodell}

Korpora bestehen meistens aus den erhobenen Primärdaten und zusätzlichen, manuell oder automatisch erstellten, Annotationen. Diese gehören meist nicht zu den Primärdaten und werden nachträglich automatisch, halbautomatisch oder manuell hinzugefügt. Hierbei kann es sich um Tokenannotationen, Spannenannotationen zur Beschreibung von Überdeckungsrelationen, Syntaxbäume für Dominanzbeziehungen oder Pointing Relations zur Beschreibungen von semantischen Beziehungen handeln. Zusätzlich werden häufig Metadaten (z.B. über Autor und Erstellungszeitpunkt) zu den Texten abgelegt.

Sehr häufig werden Wortart (Part of Speech) und Basisform (Lemma) mit Hilfe von spezialisierter Tagger-Software annotiert \citep[vgl.  z.B.][für die automatische Annotation von Wortarten]{schmidXXXXtagging}. Zusätzlich werden abhängig von den Forschungsfragen, die mit dem Korpus bearbeitet werden sollen, spezielle Annotationen erstellt (\citet[][]{carletta2005nite}, \citet[][]{woerner2006}, \citet[][]{chiarcosXXXXframework}). Hierbei kann es sich beispielsweise um Syntaxbäume über den Textdaten, um Annotationen der Informationsstruktur oder im Fall von Lernerkorpora um die Markierung und Klassifizierung von sprachlichen Fehlern handeln \citep[][]{luedeling2005lernercorpora}.

Zur Illustration unterschiedlicher Annotationsebenen: Abbildung \ref{fig:tiger-example1}, entnommen aus \citet[][]{lezius02}.

\begin{verbatim}Ein Mann kommt, der lacht.\end{verbatim}

Die Annotationsebene direkt unter dem Ausgangssatz zeigt die automatisch erstellte Ebene der Wortarten. Darunter ist eine genauere Klassifikation der Token nach ihrem Beugungsstatus zu sehen (morphologische Annotation). Über dem Text befindet sich die grafische Repräsentation der syntaktischen Struktur des Satzes. Gerade bei syntaktischen Annotationen wird deutlich, dass es sich bei den unterschiedlichen grammatischen Annotationen immer um eine Interpretation der Textoberfläche handelt. Die Einflüsse dieses Interpretationsprozesses müssen später bei der Korpusnutzung berücksichtigt werden. Dies ist jedoch ein prinzipielles Problem, auf das im Folgenden nicht weiter eingegangen wird.

\begin{figure}[H]
	\centering
	\includegraphics*[width=1.0\textwidth]{figures/tiger-example1}
	\caption{Annotierter Beispielsatz aus dem \emph{TIGER}-Korpus}
	\label{fig:tiger-example1}
\end{figure}

Die geeignete Repräsentation der unterschiedlichen Informationen (durch die Annotationen) hat große Bedeutung für deren Nachhaltigkeit und Verwertbarkeit. Im Rahmen von \emph{ANNIS 2.0} steht hier insbesondere der effiziente Austausch dieser Informationen im Vordergrund. Dieser soll im folgenden Abschnitt ausführlicher besprochen werden.

\subsection{Datenformat}\label{sec:Korpussuche.Datenformat}

Die Anforderungen an die physikalische Repräsentation von Korpusdaten kann prinzipiell an zwei wesentlichen Zielen ausgerichtet sein. Diese sind der \emph{effiziente Transport von Informationen} und die \emph{nachhaltige Speicherung von Informationen}. Die offensichtliche Gemeinsamkeit zwischen diesen beiden Zielen ist, dass die Informationen vollständig repräsentiert werden müssen. Die Anforderungen unterscheiden sich aber hinsichtlich der Optimierung des benötigten Platzes bei der Speicherung bzw. der dafür benötigten Übertragungszeit. Während es bei nachhaltigen Formaten auf die strikte Standardisierung der verfügbaren Daten ankommt \citep[][]{lehmbergXXXXstandards}, stehen bei reinen Transportformaten die Optimierung des Platz- und Bandbreitenbedarfs sowie die ressourcensparende Weiterverarbeitung im Vordergrund. \emph{ANNIS 2.0} verwendet \emph{PAULA-Inline} als internes Transportformat. 

\subsubsection{\emph{PAULA-Inline}}\label{sec:Korpussuche.Datenformat.PAULA-Inline}

 Das XML-basierte Datenformat \emph{PAULA} wurde bereits von \cite{dipper2005paula} in der \emph{stand-off} Variante vorgestellt. Stand-off bedeutet, dass Primärdaten und Annotationen in separaten Dateien vorgehalten werden (vgl. \citet[][]{thompson1997standoff} \& \citet[][]{carletta2005nite}). Weil es mit dem \emph{stand-off}-XML-Format die möglich ist, weitere Annotationsebenen an bestehende Anzufügen, ohne die bestehenden Korpus-Dateien zu berühren, sehen \cite{sperberg1994teip3} darin das notwendige Format für die Speicherung von den Daten für Mehrebenen-Korpora. Darüber hinaus ist es möglich, darin konfligierende Hierarchien abzubilden.

\emph{PAULA-stand-off} ist ein geeignetes Format für die nachhaltige Ablage von Korpusdaten und den Import in das Suchsystem \emph{DDDQuery}. Als Austauschformat zwischen dem \emph{ANNIS-Service} und der Web-Applikation ist es jedoch nur bedingt geeignet. Hier ist es wichtig, dass die Informationen in nur einem Datenstrom übertragen werden. Ebenso ist das Generieren von komplexen \emph{stand-off}-Paketen für temporäre Zwecke zu aufwendig. Deswegen wird für die interne Kommunikation im System \emph{ANNIS} die Variante \emph{PAULA-Inline} verwendet.

Eine grundlegende Eigenschaft von \emph{PAULA-Inline} ist die Vereinigung der Primärdaten und aller darüber verfügbaren Annotationen in einer einzigen XML-Datei. Dabei müssen diskontinuierliche\footnote{Eine Annotation ist diskontinuierlich, wenn sie sich auf eine unterbrochene Sequenz von Token bezieht.} Annotationen, kreuzende Kanten und Pointing Relations (siehe Abschnitt \ref{sec:Pointing Relations}) gesondert behandelt werden.

Das Textfragment ''auch zum Tragen, wenn man sich'' aus Abbildung \ref{fig:Syntaxbaum.PAULA}, inklusive der syntaktischen- und Wortartenannotation, soll in diesem Abschnitt als Beispiel für die Erläuterung der Darstellung von den drei Annotationstypen in \emph{PAULA-Inline} dienen.

\begin{figure}[H]
	\centering
	\includegraphics*[width=0.9\textwidth]{figures/DA/Syntaxbaum-Paula.pdf}
	\caption{Beispiel Syntaxbaum aus dem \emph{TIGER}-Korpus mit Wortarten nach dem STTS Tagset \citep[][]{schiller1995stts} in Klammern zur Illustration von \emph{PAULA-Inline}}\label{fig:Syntaxbaum.PAULA}
\end{figure}

In \emph{ANNIS 2.0} werden die Ergebnisse intern mit \emph{PAULA-Inline} repräsentiert und übertragen. Damit ist es möglich, alle Annotationstypen zu visualisieren. Nachfolgend werden diese eingeführt und deren Repräsentation via \emph{PAULA-Inline} vertieft.

\subsubsection{Tokenannotationen}

Nahezu alle Korpora enthalten Annotationen auf Tokenebene wie z.B. Wortart, Basisform (Lemma) oder morphologische Annotationen. Der Grund dafür ist, dass diese i.A. automatisch oder halbautomatisch erstellt werden können. Darüber hinaus bieten sie Benutzern oft wichtige Informationen für die schnelle Einordnung und Bewertung von Suchresultaten.

\newpage
\lstset{language=XML,
commentstyle=\itshape\color{Comment},
keywordstyle=\bfseries\color{Keyword},
stringstyle=\color{String},
basicstyle=\footnotesize,numbers=left,captionpos=b,frame=ltrb,framesep=5pt,caption=In \emph{PAULA-Inline} werden Tokenannotationen als Attribute auf den tok-Elementen gespeichert, label=lst:Paula.Inline.Token}

\begin{lstlisting}
<tok id="3" pos="ADV">auch</tok>
<tok id="6" pos="APPRART">zum</tok>
<tok id="7" pos="NN">Tragen</tok>
<tok id="8" pos="$,">,</tok>
<tok id="9" pos="KOUS">wenn</tok>
<tok id="10" pos="PIS">man</tok>
<tok id="11" pos="PRF">sich</tok>
\end{lstlisting}

Listing \ref{lst:Paula.Inline.Token} zeigt, wie die einzelnen Tokenannotationen des Beispiels aus Abschnitt \ref{sec:Korpussuche.Datenformat.PAULA-Inline} in \emph{PAULA-Inline} als XML-Attribute auf den \emph{tok}-Elementen gespeichert werden. Nach den Tokenannotationen treten Annotationen über Wortspannen, die sog. Spans, als zweithäufigste Art auf.

\subsubsection{Spans}

Spans sind Annotationen, die sich auf eine Sequenz von Token beziehen. Diese Sequenz kann auch diskontinuierlich sein. In dem Beispiel werden die Token ''kommt [...] zum tragen'' von einer Beispiel-Annotation (example) näher charakterisiert.

Diese Struktur kann nicht über reines \emph{XML}-Nesting\footnote{Nesting bezeichnet die verschachtelte Einbettung von Tags in \emph{XML}.} abgebildet werden. Deswegen wird der XML-Knoten der Beispiel-Annotation bei der wiederholten Nennung referenziert.

\newpage
\lstset{language=XML,basicstyle=\footnotesize,
commentstyle=\itshape\color{Comment},
keywordstyle=\bfseries\color{Keyword},
stringstyle=\color{String},numbers=left,captionpos=b,frame=ltrb,framesep=5pt,caption=\emph{PAULA-Inline}-Repräsentation des Syntaxbaums von Abbildung \ref{fig:Syntaxbaum.PAULA} mit zwei beispielhaften Spans, label=lst:Paula.Inline}

\lstinputlisting{figures/DA/Paula-Inline.xml}

Die korrespondierende \emph{PAULA-Inline}-Repräsentation ist in Listing \ref{lst:Paula.Inline} zu sehen. Der Beispiel-Knoten wird über das Attribut \emph{\_gid} einer separaten Gruppe zugeordnet. Diese wird dann bei der Wiedererwähnung in Zeile 11 durch den \emph{Group Identifier gid} 10 referenziert.

Spans können ausschließlich Token annotieren. Für den Fall, dass sich Annotationen der gleichen Ebene dennoch überschneiden, definieren diese eine Baumstruktur. Ob es sich bei einer Annotationsebene um die Beschreibung von Überdeckungs- oder Dominanzbeziehungen handelt, kann anhand der \emph{PAULA-Inline}-Struktur i.A. nicht bestimmt werden.

\subsubsection{Syntaxbäume}

Syntaxbäume sind eine leicht verständliche Methode zur Darstellung der Strukturbildung von Sätzen. Sie ermöglichen die einfache strukturelle Analyse von Äußerungen und sind in einigen Fällen ein wichtiges Hilfsmittel für die Bedeutungserschließung. Abbildung \ref{fig:SyntaxbaumBuffalo} zeigt dies auf dramatische Weise am Beispiel des folgenden, von Rapaport 1976 \citep[vgl.][]{rapaport2006buffallo} vorgestellten Satzes, wie unverzichtbar Syntaxbäume für das Textverständnis sein können.

\begin{verbatim}
 Buffalo buffalo Buffalo buffalo buffalo buffalo Buffalo buffalo.
\end{verbatim}

\begin{figure}[H]
	\centering
	\includegraphics*[width=0.9\textwidth]{figures/DA/SyntaxbaumBuffalo.pdf}
	\caption{Syntaxbaum des englischen Satzes ''Buffalo buffalo Buffalo buffalo buffalo buffalo Buffalo buffalo.'' \citep[nach][]{rapaport2006buffallo}}\label{fig:SyntaxbaumBuffalo}
\end{figure}

\newpage
Die Bedeutung des Satzes lässt sich mit den folgenden drei Schritten erschließen:

\begin{enumerate}
	\item{[Those] (Buffalo buffalo) [whom] (Buffalo buffalo buffalo) buffalo (Buffalo buffalo).}
	\item{[Those] buffalo(es) from Buffalo [that are intimidated by] buffalo(es) from Buffalo intimidate buffalo(es) from Buffalo.}
	\item{Bison from Buffalo, New York who are intimidated by other bison in their community also happen to intimidate other bison in their community.}
\end{enumerate}

Die Darstellung von Syntaxbäumen ist nur für Korpora mit syntaktischen Annotationen möglich. Dazu zählen im Deutschen das \emph{TIGER}-Korpus \citep[vgl.][]{lezius02} und das \emph{Potsdam Commentary Corpus PCC} \citep[vgl.][]{stede2004pcc}. Der Syntaxbaum aus dem Beispiel \ref{fig:Syntaxbaum.PAULA} wird in der \emph{PAULA-Inline} Repräsentation (Listing \ref{lst:Paula.Inline}) durch die Knoten 11, 12 und 13 aufgespannt.



\subsubsection{Pointing Relations}\label{sec:Pointing Relations}

Pointing Relations werden zur Beschreibung von Beziehungen zwischen einzelnen Annotationen oder Elementen in Korpora eingesetzt. Hierbei kann es sich z.B. um sekundäre Kanten in Syntaxbäumen oder semantische Beziehungen innerhalb der Primärdaten (z.B. Koreferenz) handeln. Das folgende Textbeispiel aus Listing \ref{lst:PointingRelationsExample} soll das anhand einer Anapher-Antezedenz-Struktur, also der Wiedererwähnung eines Diskursreferenten\footnote{Als Diskursreferent bezeichnet man eine, in einem Text erwähntes, Entität. Dabei kann es sich um eine Person oder einen Gegenstand handeln.} innerhalb eines Textes, verdeutlichen.

\newpage
\lstset{language=XML,basicstyle=\footnotesize,
commentstyle=\itshape\color{Comment},
keywordstyle=\bfseries\color{Keyword},
stringstyle=\color{String},numbers=left,captionpos=b,frame=ltrb,framesep=5pt,caption=\emph{PAULA-Inline}-Beispiel einer Pointing Relation, label=lst:PointingRelationsExample}

\lstinputlisting{figures/DA/PointingRelationsExample.xml}

Die Token ''Dagmar Ziegler'' sind mit der \emph{MMAX}-Annotation\footnote{\emph{MMAX} ist ein Werkzeug zur Annotation von Koreferenz: http://www.eml-research.de} $4711$ als neuer Diskursreferent annotiert. Die Annotation $4712$ annotiert für  das Token ''sie'', dass es sich auf den Diskursreferenten $4711$ (''Dagmar Ziegler'') bezieht, da beide auf die selbe Entität referenzieren.

\subsection{Anfragesprache}\label{sec:Korpussuche.Anfragesprache}

Die Anfragesprache stellt die Schnittstelle zwischen den Suchanfragen des Benutzers und den letztendlichen Fundstellen her. Je kompetenter die Benutzer im Umgang mit der Anfragesprache sind, desto mehr entsprechen die Suchergebnisse dem eigentlich Gesuchten. Daraus ergibt sich die folgende Grundanforderung an Anfragesprachen:

\emph{Die Anfragesprache muss es den Suchenden ermöglichen, die Eigenschaften der Fundstellen so umfassend und so präzise wie möglich beschreiben zu können.}

Im Rahmen der Entwicklung von \emph{ANNIS} wurde zu diesem Zweck eine spezialisierte Anfragesprache vorgestellt und implementiert.

\subsubsection{ANNIS Query Language}\label{sec:AQL}

Die \emph{ANNIS Query Language AQL} wurde von \cite{dipper2004isis} entworfen, um die Konstruktion von komplexen Anfragen mit regulären Ausdrücken, Gruppierungen, Disjunktionen, Konjunktionen, Negationen, Einschränkungen bzw. Relationen zwischen den Knoten innerhalb von Bäumen, Spannenannotationen oder semantischen Verknüpfungen formulieren zu können.

\emph{AQL} basiert auf der Graphenrepräsentation der Korpusdaten in \emph{ANNIS}. Dabei werden sowohl Token als auch Annotationen als Knoten abgebildet und über Kanten verknüpft. Die Eigenschaften der gesuchten Fundstellen lassen sich über diese Elemente mit Hilfe von \emph{AQL} beschreiben. Die Anfrage wird dann, wie am Anfang des Abschnitts \ref{sec:ANNIS 2.0} beschrieben, von \emph{ANNIS} ausgeführt und das Suchergebnis berechnet.

Die Grundelemente von \emph{AQL} sind die \emph{Positionseigenschaften} zur Beschreibung von gesuchten Token oder Annotationen und \emph{Relationensoperatoren}, um deren Beziehungen zu charakterisieren. \emph{Positionseigenschaften} können wie folgt aussehen.

\begin{description}
	\item[Exakte Suche nach Zeichenketten: ''Haus'']{Mit der Angabe eines einfachen Suchbegriffs werden alle verfügbaren Attribute von Annotations- und Tokenelementen durchsucht. Werden Suchbegriffe in Anführungsstrichen gesetzt, führt das System eine exakte Suche durch, wobei auch Groß- und Kleinschreibung berücksichtigt wird.}
	\item[Suche mit \emph{Regulären Ausdrücken}: /H.+s/]{Suchbegriffe in Slashes werden als \emph{Reguläre Ausdrücke}\footnote{\emph{Reguläre Ausdrücke} dienen zur Beschreibung von Mustern in Zeichenketten. Sie sind ein vielseitiges Werkzeug für die Suche in Texten.} behandelt.}
	\item[Suche nach Attributwerten: pos=''NN'']{Die Suche kann auf die Werte bestimmter Attribute eingeschränkt werden. Dies geschieht durch die Eingabe des Attributnamens, gefolgt von einem Gleichheitszeichen $=$ und dem Suchbegriff in Anführungszeichen (exakte Suche) oder Slashes (Suche mit \emph{Regulärem Ausdruck})}.
\end{description}

\newpage
Jede der in einer Suchanfrage angegebenen \emph{Positionseigenschaften} bezieht sich immer auf genau ein Token- oder Annotationselement im Korpus. Sie können durch \emph{logische Operatoren} \& (UND ES GIBT) und | (ODER) \^{} (exklusives ODER) und \emph{Relationsoperatoren} zu komplexen Suchen verknüpft werden. Die bedeutendsten \emph{Relationsoperatoren} sind im Folgenden aufgelistet. Eine vollständige Beschreibung kann \cite{rosenfeldXXXX} entnommen werden.

\begin{description}
	\item[Direkte Dominanz: a > b]{Findet nur Stellen, in denen das Element $b$ von Element $a$ direkt dominiert wird.}
	\item[Indirekte Dominanz: a >* b]{Findet nur Stellen, in denen das Element $b$ von Element $a$ direkt oder indirekt dominiert wird.}
	\item[Geschwisterknoten: a \$ b]{Findet nur Stellen, in denen die Elemente $a$ und $b$ Geschwisterknoten sind. Dies bedeutet, dass ein Knoten existiert, der sowohl $a$ als auch $b$ direkt dominiert.}
	\item[Direkte Präzedenz: a . b]{Findet nur Stellen, in denen das Element $b$ in Leserichtung direkt nach dem Element $a$ folgt. Dies gilt auch für Annotationselemente (z.B. Nicht-Terminale in Syntaxbäumen). In diesem Fall wird überprüft ob der letzte Nachfahren-Knoten ganz rechts von $a$ in direkter Präzedenz zum letzten Nachfahren-Knoten ganz links von $b$ steht.
}	\item[Indirekte Präzedenz: a .* b]{Findet nur Stellen, in denen das Element $b$ in Leserichtung direkt oder indirekt nach Element $a$ folgt. Dies gilt auch, analog zur direkten Präzedenz, für Annotationselemente (z.B. Nicht-Terminale in Syntaxbäumen).}
	\item[Überdeckung: a \_=\_ b]{Findet nur Stellen, an denen sich Elemente $a$ und $b$ vollständig überdecken, d.h. $a$ und $b$ beziehen sich auf die selbe Sequenz von Token.}
	\item[Letzter äußerer Nachfahren-Knoten: a \_l\_ b bzw. a \_r\_ b]{Findet nur Stellen, an denen das Element $b$ Nachfahren-Knoten mit maximalem Abstand von $a$ ist und $b$ am weitesten links (a \_l\_ b) bzw. rechts (a \_r\_ b) steht.}
	
	\newpage
	\item[Inklusion: a \_i\_ b]{Findet nur Stellen, an denen das Element $b$ vollständig vom Element $a$ überdeckt wird. Die gilt für Token, wenn $a$ und $b$ identisch sind und bei Annotationselementen, falls $a$ mindestens die gleichen Token annotiert wie $b$.}
\end{description}

\emph{Positionseigenschaften} werden bei der Verwendung für die \emph{Relationsoperatoren} über ein einfaches Nummerierungssytem referenziert. Dafür werden alle in der Suchanfrage vorhandenen \emph{Positionseigenschaften} mit 1 beginnend durchnummeriert. Mit dem Operator \# kann dann auf diese verwiesen werden. Abschließend folgt eine Auflistung von Beispielanfragen und deren verbale Beschreibung.

\begin{description}
	\item{\textbf{tok=/.*trock.*/}\ Tokensuche\\ \emph{trocken, trockner, getrocknet, auszutrocknen, Wäschetrockner, …}}
	
	\item{\textbf{tok=/[A-ZÄÖÜ][a-zäöüß]+-[A-ZÄÖÜ][a-zäöüß]+/}\\ Findet Bindestrich-Komposita\\\emph{Radio-Nachrichten, Mikro-Variation, Nibelungenlied-Handschriften, ...}}
	
	\item{\textbf{lemma=/.+[\^{ }aeiouäöü]chen/ \&\ pos="NN" \&\ \#1\ \_=\_\ \#2}\\ Verkleinerungsformen\\ \emph{Bändchen, Kästchen, Häschen, ...}}
	
	\item{\textbf{lemma=/[\^{ }äöü]+/ \& tok=/.+[äöü].+/\ \& pos="NN"\ \& \#1\ \_=\_\ \#2\ \&\ \#2\ \_=\_\ \#3}\\ Findet umgelautete Pluralformen\\ \emph{Züge, Fragesätze, Entwürfe, ...}}
	
	\item{\textbf{pos="KOUS"\ \&\ "man"\ \&\ "{}sich"\ \&\ \#1 . \#2 \& \#2 . \#3}\\ Nebensätze mit Subjekt "man"{} und einem Reflexivum\\ \emph{weil man sich\\ ob man sich}}
	
	\newpage
	\item{\textbf{cat="{}S"\ \&\ pos="PTKVZ"\ \&\ pos="VVFIN"\ \&\ \#1 > \#3\ \&\ \#2\ \_l\_\ \#1}\\ Verbzweit-Sätze mit vorangestellter Verbpartikel\\ \emph{\textbf{Verloren} ging dabei endgültig das Selbstverständnis der Einheimischen}\\ 
	\emph{\textbf{Fest} steht, dass dort 580 der insgesamt 4640 Arbeitsplätze wegfallen}
	}
	
	\item{\textbf{cat="{}S"\ \&\ pos="VVFIN"\ \&\ cat=/.*/\ \&\ cat=/.*/\ \&\ \#1\ >"HD"\ \#2\ \&\ \#1\ \_l\_\ \#3\ \&\ \#3\ .\ \#4\ \&\ \#4 .\ \#2}\\ Mehrfache Vorfeldbesetzung nach \cite{muellerXXXXsyntax}\\ 
	\emph{[Am schwersten] [mit der Selbstkritik] tat sich Jürgen Kocka.}\\ 
	\emph{[Negativ] [auf den Gewinn] wirkten sich vor allem Wechselkursschwankungen aus.}
	}
	
	\item{\textbf{cat="NP"\ \&\ cat=/(NP|PP)/\ \&\ pos=/(VVFIN|VVINF)/\ \&\ pos="PRELS"\ \&\ tok\ \&\ \#2\ > \#1\ \&\ \#2\ > \#4\ \&\ \#5\ \_l\_\ \#1\ \&\ \#1\ .*\ \#3\ \&\ \#3\ .*\ \#4}\\ Extraposition eines Relativsatzes, der eine Nominalphrase innerhalb von einer PP oder NP modifiziert, über die Phrasengrenzen hinweg\\
	\emph{Der 43jährige will nach eigener Darstellung damit [NP den Weg [PP für [NP eine Diskussion [PP über [NP den künftigen Kurs [NP der stärksten Oppositionsgruppierung]]]]]] freimachen, [die aber mit 10,4 Prozent der Stimmen bei der Wahl im Oktober weit hinter den Erwartungen zurückgeblieben war].}
	}
\end{description}

\newpage
\subsubsection{Stärken und Schwächen von \emph{AQL}}\label{sec:AQL.Usability}

Durch die direkte Projektion der Anfragesprache \emph{AQL} auf die zugrunde liegende Graphenstruktur der Korpusdaten (Knoten und Kanten) ergeben sich deutliche Vorteile für die Vermittlung und Verwendung bei der Korpussuche. Für das Erlernen der Korpussuche muss hauptsächlich die grundlegende Repräsentation der Korpusinformationen verstanden werden. Die eigentliche Suchanfrage ist somit die Formulierung eines gesuchten Musters in den verfügbaren Daten \citep[vgl.][]{wynneXXXXsearching}. \emph{AQL} ermöglicht dann die direkte Übersetzung dieser Suchmuster in eine Suchanfrage.

Aber auch bei diesem Übersetzungsprozess können Fehler auftreten. Durch die Trennung von Überdeckungsbeziehungen (Spans), Dominanzbeziehungen (Syntaxbäume) und semantischen Beziehungen (Pointing Relations) im Datenmodell ist den Benutzern nicht immer klar, wie nach einem bestimmten Phänomen zu suchen ist. Das wird insbesondere dadurch verstärkt, dass man eine verbalisierte Suchanfrage in unterschiedliche, für die jeweilige Anfrage und ein bestimmtes Korpus äquivalente, \emph{AQL}-Anfragen übersetzen kann. So kann z.B. die Suche nach ''Adjektiven (ADJA), die sich in einer Nominalphrase befinden.'' in einigen Korpora auf zwei Arten formuliert werden. 

\begin{enumerate}
	\item{\textbf{pos=''ADJA'' \&\ tiger:cat=''NP''\ \& \#2\ \_i\_\ \#1}}
	\item{\textbf{pos=''ADJA'' \&\ tiger:cat=''NP''\ \& \#2\ > \#1}}
\end{enumerate}

Die Unterscheidung dieser beiden Anfragen geschieht ausschließlich durch die Verwendung unterschiedlicher Relationsoperatoren (\_i\_ vs. >), die in dem Fall die gleichen Fundstellen produzieren. Die Anfragen sind hinsichtlich der verbalisierten Suchanfrage in einigen Korpora äquivalent, weil die Datenbasis entsprechend erstellt wurde. Für ein anderes Korpus muss das nicht gelten.

Deswegen muss der Benutzer hier ausdrücklich auf die jeweilige Beschreibung der Korpora hingewiesen werden, um unnötige Fehler zu vermeiden. Es ist aber eher eine Frage der einheitlichen Repräsentation unterschiedlicher Korpora in einem System als ein spezifisches Problem von \emph{AQL}.

Eine mögliche Optimierung von \emph{AQL} besteht jedoch darin, eine freie Benennung von Referenzen zuzulassen. Dies würde bedeuten, dass Teile einer Anfrage nicht nur über eine Nummer (z.B. \#1), sondern auch über einen zuvor vergebenen Namen (z.B. \#meinAdjektiv) benannt werden können. Dies würde zwar eine Änderung der Anfragesprache und damit der verfügbaren Compiler notwendig machen, aber gerade die Lesbarkeit langer Suchausdrücke würde sich dadurch erheblich verbessern lassen.

Nachdem nun auch die Möglichkeiten von \emph{AQL} vorgestellt wurden, muss der letzte wichtige Aspekt von komplexen Korpussuchsystemen, die Benutzerschnittstelle, näher betrachtet werden.

\subsection{Benutzerschnittstelle}\label{sec:Korpussuche.Benutzerschnittstelle}

Wie auch in anderen komplexen Anwendungen kann die Benutzerschnittstelle als das Gesicht von \emph{ANNIS 2.0} verstanden werden. Über sie interagieren die Benutzer mit den Funktionen des Systems. Sie zeigt verfügbare Korpora mit allen notwendigen Metadaten an, nimmt Suchanfragen entgegen und stellt die Ergebnisse in Listen dar. Für viele Benutzer wird sie ein Synonym für das Gesamtsystem werden.

Neben den vielen expliziten Anforderungen für \emph{ANNIS 2.0} sind die Erfahrungen, die mit der Benutzerschnittstelle von \emph{ANNIS 1.0} gesammelt werden konnten, von großer Bedeutung. Hauptkritikpunkte sind hier die unübersichtliche Darstellung von Ergebnislisten und die Darstellung der Annotationsebenen. Diese werden nämlich einheitlich in tabellarischer Form angezeigt. Besonders für Syntaxbäume ist diese Darstellung unzureichend.

\begin{figure}[H]
	\centering
	\includegraphics*[width=0.75\textwidth]{figures/DA/AnnisSyntaxTab.jpg}
	\caption{\emph{ANNIS 1.0} stellt Syntaxannotationen in HTML-Tabellen dar.}
	\label{fig:Syntaxbaum.ANNIS1}
\end{figure}

Abbildung \ref{fig:Syntaxbaum.ANNIS1} zeigt wie Syntaxbäume in \emph{ANNIS 1.0} dargestellt werden. Es wird deutlich, dass mit dieser Methode die Informationen nur schwer vermittelt werden können. Eine grafische Ansicht der Annotationen ist hier besser geeignet.

Dieses Beispiel zeigt, dass Systeme zur Korpussuche nicht nur über deren Suchgeschwindigkeit, die Mächtigkeit der verwendeten Anfragesprache oder andere technische Kriterien bewertet werden können. Die Benutzerschnittstelle trägt maßgeblich zur Benutzbarkeit des Systems und damit zur die Zufriedenheit der Benutzer bei. Im Gegensatz zu anderen Komponenten gelten dafür aber neben technischen Anforderungen auch Anforderungen an die Anwenderfreundlichkeit. Diese werden im nächsten Kapitel betrachtet.

\newpage
\section{Theoretische Vorüberlegungen}\label{sec:Usability}

\emph{Benutzerfreundlichkeit} (engl. \emph{ease of use}) ist ein Begriff, der oft zu Werbezwecken verwendet wird. Als Kriterium für die Bewertung und Entwicklung von interaktiver Software ist er jedoch zu unscharf. An dieser Stelle können \emph{Normen} mit ihren Beispielen und Erläuterungen einheitliche Terminologien für die Beschreibung von Projektzielen schaffen und darüber hinaus konkrete Anforderungen formulieren.

Normen sind die Ergebnisse softwareergonomischer Betrachtungen und schließen die wichtigsten Aspekte der Mensch-Computer-Interaktion ein. Dazu gehören die Erwartungen von Anwendern an die Software, die Betrachtung der Vorkenntnisse der Anwender und deren Lernbereitschaft. Aber auch die organisatorischen Rahmenbedingungen in denen der Anwender seine Aufgaben erfüllen muss, werden mit einbezogen.

Dieses Kapitel stellt in den folgenden Abschnitten grundlegende Normen (Abschnitt \ref{sec:Normen}) und Heuristiken (Abschnitt \ref{sec:Heuristiken}) für die Gestaltung von benutzerfreundlicher Software vor und konkretisiert diese anhand der durchgeführten Nutzerbefragung und dem Entwurf von archetypischen Benutzerprofilen (Abschnitt \ref{sec:Personas}). In Kapitel \ref{sec:ANNIS 2.0} wird dann die konkrete Umsetzung dieser Gestaltungsprinzipen für \emph{ANNIS 2.0} diskutiert.

\subsection{DIN EN ISO 9241}\label{sec:Normen}

Die einschlägige deutsche Industrienorm DIN\footnote{DIN: Deutsche Industrienorm} EN\footnote{EN: European Norm} ISO\footnote{ISO: International Standardisation Organisation} 9241 hat den Titel ''Ergonomische Anforderungen für Bürotätigkeiten mit Bildschirmgeräten'' und geht in wesentlichen Punkten aus der nicht mehr gültigen DIN 66234 hervor. Sie besteht derzeit aus 17 Teilen, die Anforderungen an Hardware und Dialoggestaltung von Software formulieren.

Für die Entwicklung einer Benutzerschnittstelle fassen die ''Anforderungen an die Gebrauchstauglichkeit'' (Teil 11) und die ''Grundlagen der Dialoggestaltung'' (Teil 10) wichtige Gestaltungsprinzipien zusammen. Der Teil 11 formuliert die grundlegenden Prinzipien zur Bewertung von Software.

\subsubsection{Anforderungen an die \emph{Gebrauchstauglichkeit} - Teil 11}

Um Software nach ergonomischen Gesichtspunkten so objektiv wie möglich bewerten zu können, werden wohldefinierte Bewertungskriterien benötigt. Teil 11 der DIN 9241 formuliert drei übergeordnete Normkriterien, aus denen sich das Maß der \emph{Gebrauchstauglichkeit} ableiten lässt. 

\begin{description}
	\item[Effektivität]{bezeichnet die Genauigkeit und Vollständigkeit, mit der die Benutzer ein bestimmtes Ziel erreichen.}
	\item[Effizienz]{bezeichnet den Aufwand der Benutzer im Verhältnis zur Genauigkeit und Vollständigkeit des erzielten Effekts.}
	\item[Zufriedenheit]{bezeichnet die positive Einstellung der Benutzer gegenüber der Nutzung des Systems sowie ihre empfundene Freiheit oder Beeinträchtigung durch das System.}
\end{description}

Diese Kriterien sind nicht operationalisierbar oder technisch messbar. Sie orientieren sich an den Benutzern und deren Arbeitsaufgaben. Sie sind die Grundlage der Definition für das Maß der \emph{Gebrauchstauglichkeit}.

\begin{description}
	\item[Gebrauchstauglichkeit (engl.  \emph{Usability})]{ist das Maß der \emph{Effektivität}, \emph{Effizienz} und \emph{Zufriedenheit}, mit der Benutzer mit einem bestimmten System vorgegebene \emph{Ziele} erreichen können.}
\end{description}

Die so definierte \emph{Gebrauchstauglichkeit} ist der zentrale Begriff der Software-Ergonomie und impliziert, dass die Benutzbarkeit von Software nur innerhalb eines definierten Nutzungskontextes der auch die Aufgaben und Ziele der Benutzung einschließt bewertet werden kann. Ziele (siehe Abschnitt \ref{sec:Korpussuche}) müssen klar definiert und, so weit es geht, überprüfbar sein. Auf unscharfe Ziele wie sie vorwiegend bei der Benutzung von Unterhaltungselektronik oder Spiele-Software existieren, wird in dieser Arbeit nicht näher eingegangen.

Beim Entwurf einer Benutzerschnittstelle für die Korpussuche ist es in besonderem Maße wichtig, dass der Benutzer darin unterstützt wird, die richtigen Suchanfragen einzugeben und die Ergebnisse möglichst effizient auszuwerten. Diese Aufgaben werden ausschliesslich von der Benutzerschnittstelle erfüllt. Deshalb ist es sinnvoll, eine anerkannte Zusammenstellung von Entwurfsrichtlinien mit in den Gestaltungsprozess einzubeziehen.

\subsubsection{Grundlagen der Dialoggestaltung - Teil 10}\label{sec:DIN9241-10}

In Teil 10 der DIN EN ISO 9241 werden die notwendigen Eigenschaften eines Dialogsystems formuliert. Da es sich bei jeder interaktiven Software um ein Dialogsystem handelt, sind diese Kriterien auf fast jede Art von Benutzerschnittstelle anwendbar und erstrebenswert.

Die sieben Normkriterien, welche von Dialogsystemen mit hoher \emph{Gebrauchstauglichkeit} erfüllt werden müssen, sind nachfolgend aufgelistet und nach \citet[][]{dahm06} erläutert.

\begin{description}
	\item[Aufgabenangemessenheit]{Der Benutzer soll bei der Erledigung seiner Arbeitsaufgaben unterstützt werden.}
	\item[Selbstbeschreibungsfähigkeit]{Jeder einzelne Dialogschritt ist durch Beschreibungen oder Rückmeldungen unmittelbar verständlich oder er wird auf Anfrage des Benutzers erklärt.}
	\item[Steuerbarkeit]{Der Benutzer soll in der Lage sein, den Dialogablauf zu steuern, das heißt den Ablauf, Richtung und Geschwindigkeit beeinflussen zu können, bis er sein Ziel erreicht hat.}
	\item[Erwartungskonformität]{Der Dialog entspricht den Kenntnissen des Benutzers aus seinem Arbeitsgebiet, resultierend seiner Ausbildung oder seiner Erfahrung. Außerdem ist der Dialog konsistent.}
	\item[Fehlertoleranz]{Trotz erkennbar fehlerhafter Eingaben kann das beabsichtigte Arbeitsergebnis mit keinem oder minimalem Korrekturaufwand des Benutzers erreicht werden.}
	\item[Individualisierbarkeit]{Der Benutzer kann den Dialog an seine Arbeitsaufgabe sowie seine individuellen Fähigkeiten und Vorlieben anpassen.}
	\item[Lernförderlichkeit]{Der Benutzer wird beim Erlernen der Anwendung unterstützt und angeleitet.}
\end{description}

Obwohl diese Kriterien eine ausreichende Grundlage für die Entwicklung einer nutzerorientierten Suchschnittstelle bilden, sollen sie an dieser Stelle durch weitere Entwurfsprinzipien ergänzt werden.

\subsection{Die zehn Usability-Heuristiken von Nielsen}\label{sec:Heuristiken}

\citet[][]{nielsen-molich90} stellen mit der \emph{Heuristischen Evaluation} eine ingenieurspsychologische Methode zum Auffinden von Usability-Problemen in Benutzerschnittstellen vor. Sie wird von einer kleinen Gruppe von Evaluatoren durchgeführt. Diese untersuchen die Benutzerschnittstelle und bewerten sie anhand verschiedener Usability-Prinzipien. Die \emph{Heuristische Evaluation} kann gut in iterative Entwurfsmethoden, wie sie auch bei der Umsetzung von \emph{ANNIS 2.0} angewendet wurden, integriert werden.

\citet[][]{nielsen94-heuristic-eval} verfeinern diese Heuristiken, basierend auf der Faktorenanalyse von 249 Usability-Problemen. Daraus ergeben sich die zehn aussagekräftigsten Heuristiken.

\begin{description}
	\item[Rückmeldungen]{Das System sollte die Benutzer ständig darüber informieren, welche Arbeitsschritte gerade ausgeführt werden. Dies sollte durch angemessene Rückmeldungen und in vertretbarem Zeitraum geschehen.}
	
	\newpage
	\item[Ausdrucksweisen des Anwenders]{Das System sollte die Sprache der Benutzer sprechen. Terminologien und Konzepte sollen dem Benutzer bekannt sein und dessen Fachbereich entsprechen. Informationen sollten geeigneten Konventionen folgen und in logischer Reihenfolge angezeigt werden.}
	\item[Konsistenz und Standards]{Den Benutzern sollte immer klar sein, ob Aktionen, Situationen oder Terme die gleiche Bedeutung haben. Konventionen müssen strikt eingehalten werden.}
	\item[Fehlervermeidung]{Noch besser als gute Fehlermeldungen ist ein sorgfältiger Systementwurf, der Problemen vorbeugt. Dies bedeutet, dass fehleranfällige Zustände eliminiert bzw. Eingaben stets überprüft  werden und ggf. vor Ausführung der Aktion eine Bestätigung vom Benutzer angefordert wird.}
	\item[Minimale mentale Belastung des Benutzers]{Die notwendige Gedächtnisleitung des Nutzers sollte minimiert werden indem Objekte, Aktionen und Optionen stets sichtbar sind. Der Benutzer sollte sich beim Wechsel von einem Dialog in einen anderen keine Informationen  merken müssen. Instruktionen zur Benutzung des Systems sollten sichtbar sein, wenn es notwendig ist.}
	\item[Ästhetik und minimalistisches Design]{Dialoge sollten keine irrelevanten oder selten verwendeten Informationen enthalten. Jede zusätzliche Informationseinheit kann mit einer wichtigeren Information konkurrieren und deren relative Sichtbarkeit herabsetzen.}
	\item[Klare Auswege]{Benutzer aktivieren Systemfunktionen oft aus Versehen und benötigen deshalb eine klar erkennbare Abbruchmöglichkeit, um den ungewollten Zustand schnell zu verlassen, ohne einen langen Dialog zu durchlaufen. \emph{Undo} und \emph{Redo} sollten unterstützt werden.}
	\item[Abkürzungen]{
	Abkürzungen, dem Anfänger unbekannt, können oft die Interaktion mit fortgeschrittenen Benutzern beschleunigen. Das System kann so unerfahrene und erfahrene Benutzer unterstützen. Benutzer sollten häufige Aufgaben gemäß ihren Vorlieben anpassen können.}
	\item[Gute Fehlermeldungen]{Fehlermeldungen sollten in natürlicher Sprache (ohne Codes) das Problem präzise beschreiben und konstruktive Lösungsvorschläge anbieten.}
	\item[Hilfe und Dokumentation]{Auch wenn ein System selbsterklärend ist, kann es notwendig sein, Hilfe und Dokumentation anzubieten. Diese sollte einfach zu durchsuchen und an Aufgaben des Benutzers ausgerichtet sein, konkrete Ausführungsschritte enthalten und nicht zu umfangreich sein.}
\end{description}

Betrachtet man die Heuristiken von Nielsen im Vergleich mit den ''Grundlagen der Dialoggestaltung'' in DIN 9241, so werden starke Überschneidungen deutlich. Gerade aber weil es sich um unscharfe Anforderungen handelt, ist es notwendig, mehrere Quellen und Standpunkte zu betrachten und diese für die jeweilige Aufgabenstellung zu adaptieren.

Aber auch die Kombination der hier vorgestellten Betrachtungsweisen kann nur die Basis konkreter Realisierungsansätze für eine nutzerorientierte Korpussuche sein. Um besseres Verständnis über die Bedürfnisse, Vorwissen und Ziele der unterschiedlichen Benutzergruppen zu erlangen, müssen die Bedürfnisse der zu erwartenden Benutzergruppe näher betrachtet werden.

Dazu werden nun die durchgeführte Nutzerbefragung und die erfassten Antworten vorgestellt und diskutiert. Daraus werden danach in Abschnitt \ref{sec:Personas} prototypische fiktive Personen abgeleitet. Damit können dann die konkreten Anforderungen für die Umsetzung von \emph{ANNIS 2.0} formuliert werden. 

\newpage
\subsection{Die Nutzerbefragung}\label{sec:Nutzerbefragung}

Die Erfassung der Bedürfnisse, des Vorwissens und der Ziele von Benutzern stellt einen entscheidenden Schritt für die Entwicklung einer Korpussuche mit hoher \emph{Gebrauchstauglichkeit} dar. Nutzer verfolgen mit der Suche in Korpora immer ein bestimmbares Ziel (siehe Abschnitt \ref{sec:Korpussuche}). Die Ausrichtung der Benutzerschnittstelle an diesen Zielen und dem Wissen der Benutzer unterscheidet sich signifikant von der Entwicklung einer Suchschnittstelle für eine bestimmte Datenbasis. Im Fall von \emph{ANNIS} lag der Schwerpunkt auf zwei Befragungsmethoden. Diese sind das freie Interview mit Experten und Benutzern sowie ein Onlinefragebogen.

In einem freien Interview können Nutzer die Bedürfnisse und Wunschvorstellungen frei formulieren. Durch die sehr hohe Spezifität der Antworten und der geringen Vergleichbarkeit der Ergebnisse eignet sich diese Methode eher dazu, eine Menge möglicher Anforderungen zu erhalten und individuelle Tendenzen grob zu erfassen. Die Interviews hatten keinen festen Zeitrahmen und fanden meist spontan statt. Dabei wurden je nach Befragtem Teilfragen der Anwendung oder Aspekte aus der übergeordneten Problemstellung abgedeckt. Die Interviews wurden nicht aufgezeichnet oder protokolliert und finden sich deswegen nur implizit in dieser Arbeit wieder. Die aus den Gesprächen erhaltenen Informationen wurden durch eine strukturierte Onlinebefragung einer grösseren Nutzergruppe gestützt und verallgemeinert. 

%Eine Verfeinerung dieser freien Interviews wäre die Durchführung von strukturierten- und standardisierten Interviews. Das strukturierte Interview bietet durch einen Fragenkatalog eine bessere Kontrolle über den Verlauf des Einzelbefragung. Die Fragen sind hierbei offen, was die Vergleichbarkeit der Antworten auch bei dieser Methode erschwert. Das standardisierte Interview stellt eine strikte Befragungsmethode dar. Es werden sowohl Fragen, als auch feste Antwortmöglichkeiten vorgegeben. Durch den hohen erforderlichen Zeitaufwand kommt das Interview für die Befragung jedoch nicht in Frage.

\subsubsection{Durchführung}

Die Nutzerbefragung wurde in Form eines Online-Fragebogens mit der Web-Umfrage-Software \emph{LimeSurvey}\footnote{\emph{LimeSurvey} Homepage: http://limesurvey.org/} durchgeführt. Der Fragebogen beinhaltet die Erfassung von verwendeten Korpus-Typen, der Erfahrung mit der Annotation von Texten und der Selbsteinschätzung der Fertigkeiten in der Korpusauswertung.

Da es sich bei dieser Untersuchung hauptsächlich um Fragen zur Selbsteinschätzung handelt, muss bei der Auswertung der \emph{Self-Serving Bias} berücksichtigt werden. Demnach ''werden Selbstbeurteilungen mit dem Selbstkonzept in Einklang gebracht und fallen eher selbstwertstützend aus'' \cite[][S. 183]{bortz2006methoden}. Objektivität kann hier nur durch die Anwendung von Wissensfragen hergestellt werden. Die Ausarbeitung eines umfassenden Wissenstests für das Gebiet der Korpuslinguistik ist jedoch so umfangreich, dass sie den Rahmen dieser Arbeit überschreitet. Um die Ergebnisse der Selbsteinschätzungen zu objektivieren werden zwei Fragen zur Suche mit \emph{regulären Ausdrücken} und zur Tokenisierung\footnote{Tokenisierung ist die Zerlegung eines Textes in atomare Einheiten (Token).} von Texten gestellt. 

Mit einer Ausnahme sind alle Fragen geschlossene Fragen mit vorgegebenen Antwortmöglichkeiten. Dabei handelt es sich sowohl um Einfach- und Mehrfachauswahlen als auch um eine Rangreihe. In einigen Fällen wird zusätzlich eine Freitext-Antwort angeboten, um eventuellen Spezialanwendungen gerecht zu werden. Der Inhalt des Fragebogens ist im Anhang \ref{sec:Fragebogen.Inhalt} vollständig aufgeführt. 

%\cite{tielsch2008wahrnehmung} (S. 95-101) diskutiert die Teilaspekte von Online-Untersuchungen. Das sind die Stichprobenziehung, Identität und Kontrolle der Befragten, Drop-out, technisch bedingte Vor- und Nachteile und ethische Aspekte.

Die Befragten wurden über E-Mails auf den Verteilerlisten der Korpuslinguistik HU-Berlin und der Liste des SFB-632 zur Teilnahme an der Befragung eingeladen. Dadurch wurde sichergestellt, dass die Befragten auch fachnah sind und ein Interesse an der Befragung haben. Nachteilig ist, dass Korpusinteressierte ohne jede Vorerfahrung ausgeschlossen werden. Ebenso sind die Listen verhältnismäßig klein und können dadurch nur wenige Teilnahmen generieren.

Alle Antworten der Befragung wurden anonymisiert, wodurch mehrfache Teilnahmen nicht identifiziert werden können. Es kann aber davon ausgegangen werden, dass dies nicht vorgekommen ist. Auch die Umstände der Beantwortung konnten nicht kontrolliert werden.

\subsubsection{Auswertung}\label{sec:Nutzerbefragung.Auswertung}

Bevor in diesem Abschnitt die Ergebnisse der Nutzerbefragung vorgestellt werden, bleibt anzumerken, dass die Gesamtbeteiligung nur sehr gering war. Insgesamt ist der Fragebogen 21 mal (davon einmal unvollständig) ausgefüllt worden. Dies bedeutet, dass die Ergebnisse ausschließlich als Richtlinie, keinesfalls aber als allgemeingültiges Werkzeug betrachtet werden dürfen. Positiv aber ist, dass viele der im Vorfeld getroffenen Annahmen zum Vorwissen der Benutzer und zu den Anforderungen an das System bestätigt wurden. Eine Tatsache, die nicht zuletzt an der Homogenität der Korpusnutzer liegt.

Tabelle \ref{tab:Frage1} zeigt die Angaben zum Vorwissenszeitraum der Benutzer zu Korpora. Mit 85\% der Angaben bei ''>3 Jahren'' ist davon auszugehen, dass bei dieser Gruppe auch erweitertes Vorwissen vorhanden ist. 

	\begin{table}[H]
		\centering
		\begin{tabular}{l | r}
					Frage 1 & \\
					\hline\\
					keine Antwort	& 5\% \\
					<  3 Monate 	& 0\% \\
					3-6 Monate	& 0\% \\
					6-9 Monate 	& 0\% \\
					9-12 Monate	& 0\% \\
					1-2 Jahre		& 5\% \\
					2-3 Jahre		& 10\% \\
					> 3 Jahre		& 85\%
		\end{tabular}
		\caption{Wie lange wissen Sie schon von Korpora?}\label{tab:Frage1}
	\end{table}

Im Vergleich zur ersten Frage zeigt sich bei den Ergebnissen aus Frage 2 (Tabelle \ref{tab:Frage2}), dass die Mehrheit der Befragten (80\%) angibt, Korpora seit mindestens  einem Jahr für Arbeiten zu verwenden. Diese Gruppe kann als ''erfahren'' angesehen werden.

	\begin{table}[H]
		\centering
		\begin{tabular}{l | r}
					Frage 2 & \\
					\hline\\
					keine Antwort	& 5\% \\
					<  3 Monate 	& 0\% \\
					3-6 Monate	& 5\% \\
					6-9 Monate 	& 5\% \\
					9-12 Monate	& 5\% \\
					1-2 Jahre		& 10\% \\
					2-3 Jahre		& 25\% \\
					> 3 Jahre		& 45\%
		\end{tabular}
		\caption{Wie lange benutzen Sie schon Korpora für Ihre Arbeit(en)?}\label{tab:Frage2}
	\end{table}

Die Antworten auf die Frage 3 nach der Bedeutung von Korpusdaten für Veröffentlichungen zeigen, dass die Mehrheit (75\%) der Befragen Korpusdaten als ''wichtig'' oder ''sehr wichtig'' für die linguistische Arbeit empfinden.

	\begin{table}[H]
		\centering
		\begin{tabular}{l | r}
					Frage 3 & \\
					\hline\\
					keine Antwort		& 5\% \\
					nicht wichtig 		& 5\% \\
					weniger wichtig	& 10\% \\
					wichtig 			& 25\% \\
					sehr wichtig		& 55\%
		\end{tabular}
		\caption{Wie wichtig ist die Auswertung von Korpusdaten für Ihre Veröffentlichungen?}\label{tab:Frage3}
	\end{table}

Frage 4 ist eine umschriebene Frage danach, ob die Befragten die Korpussuche auch lokal, d.h. ohne Netzwerkverbindung, auf dem eigenen Computer ausführen müssen. Die Angaben von über 70\% mit ''ja'' untermauern die Bedeutung einer portablen Version von \emph{ANNIS}. Die Ideen dazu werden im Ausblick (Abschnitt \ref{sec:ANNIS.local}) ausgeführt. Alle Angaben ''unsicher'' (20\%) können als ''nein'' gewertet werden, dies jedoch nur unter der Annahme, dass alle Benutzer, die lokale Suchen benötigen (z.B. bei der Feldforschung), den Bedarf kennen und auch äußern würden. Es kann jedoch nicht ausgeschlossen werden, dass einigen Teilnehmern die Fragestellung nicht klar war und sie deshalb intuitiv geantwortet haben.

	\begin{table}[H]
		\centering
		\begin{tabular}{l | r}
					Frage 4 & \\
					\hline\\
					ja 			& 75\% \\
					nein			& 5\% \\
					unsicher 		& 20\% 
		\end{tabular}
		\caption{Benötigen Sie die Möglichkeit, Korpusdaten offline (d.h. lokal auf Ihrem Computer) umfangreich auswerten zu können?}\label{tab:Frage4}
	\end{table}
	
Die Ergebnisse der folgenden Frage 5 sollten direkten Einfluss auf die zukünftige Weiterentwicklung von \emph{ANNIS} haben. Sie zeigen nämlich, dass die befragten Korpusnutzer angeben, an verschiedenartigen Korpustypen interessiert zu sein. Der daraus entstehende Bedarf an angepassten Darstellungsmethoden sollte unbedingt berücksichtigt werden. Insbesondere die Einbindung von multimedialen Annotationswerten (Audio, Video, Bild) und die Ablage und Darstellung von Parallelkorpora können von \emph{ANNIS 2.0} noch nicht vollständig unterstützt werden.

	\begin{table}[H]
		\centering
		\begin{tabular}{l | r r r r l | r}
					Frage 5 & (gar nicht) 1 & 2 & 3 & 4 & 5 (sehr) & \bf{$\varnothing$}	\\
					\hline\\
					(Synchrone) Textkorpora 			& 5\% &   5\%        & 10\%             & 25\%       & 55\% & \bf{4.20} \\
					Korpora gesprochener Sprache 	& 5\% &   10\%        & 10\%             & 45\%       & 30\% & \bf{3.90} \\
					Multimodale Korpora 			& 20\% &   25\%        & 35\%             & 10\%       & 5\% & \bf{2.53} \\
					Parallelkorpora 				& 10\% &   15\%        & 20\%             & 20\%       & 30\% & \bf{3.47} \\
					Historische Korpora 				& 40\% &   20\%        & 15\%             & 10\%       & 10\% & \bf{2.26} \\
		\end{tabular}
		\caption{Wie sehr interessieren Sie sich für folgende Korpusarten?}\label{tab:Frage5}
	\end{table}
	
Die Schlussfolgerungen aus den Angaben zu Frage 5 werden durch die Auswertung von Frage 6 (Tabelle \ref{tab:Frage6}) gestützt. Hierbei zeigt sich, dass es weder stark favorisierte noch unbedeutende Annotationsebenen gibt. 
		
	\begin{table}[H]
		\centering
		\begin{tabular}{l | r r r r r | r}
					Frage 6 & (gar nicht) 1 & 2 & 3 & 4 & 5 (sehr) & \bf{$\varnothing$}	\\
					\hline\\
					Wortarten/Lemmatisierung	& 5\% &   5\%        & 5\%             & 30\%       & 55\% & \bf{4.25} \\
					Morphologie				& 0\% &   10\%      & 15\%             & 30\%       & 40\% & \bf{4.05} \\
					Syntaktische Annotation\\ (Baumbanken)	 & 10\% &   20\%        & 10\%             & 40\%       & 20\% & \bf{3.58} \\
					Semantische Annotation		& 25\% &   20\%        & 10\%             & 30\%       & 10\% & \bf{2.79} \\
					Informationsstruktur			& 35\% &   5\%        & 15\%             & 20\%       & 15\% & \bf{2.72} \\
					Koreferenz				& 50\% &   10\%        & 20\%             & 10\%       & 5\% & \bf{2.05}
		\end{tabular}
		\caption{Wie oft nutzen Sie folgende Annotationsebenen in Ihren Auswertungen?}\label{tab:Frage6}
	\end{table}

Die Frage 7 (Tabelle \ref{tab:Frage7}) dient Feststellung, mit welchen Betriebssystemen die Benutzer arbeiten. Hierbei nehmen die Mircrosoft-Betriebssysteme Windows XP/Vista/98 die Mehrheitsposition ein. Dennoch geben jeweils ein Drittel der Benutzer an, auch Linux oder Mac OS X für die Arbeit zu verwenden. Gerade Mac OS und dem dabei oft verwendeten Browser \emph{Safari} können bei der Entwicklung von Web-Anwendungen besondere Anforderungen stellen. Um eine breite Akzeptanz zu schaffen, müssen diese jedoch berücksichtigt werden.
	
	\begin{table}[H]
		\centering
		\begin{tabular}{l | r}
			 		Frage 7 & \\
					\hline\\
					Windows 98/ME	& 10\% \\
					Windows XP 		& 57\% \\
					Windows Vista		& 10\% \\
					Linux 			& 29\% \\
					MacOS 9			& 0\% \\
					MacOS X			& 33\% \\
					Sun Solaris		& 5\% \\
					Sonstige			& 0\%
					
		\end{tabular}
		\caption{Welche(s) Betriebssystem(e) verwenden Sie für Ihre Arbeit?}\label{tab:Frage7}
	\end{table}

Die Frage 8 (Tabelle \ref{tab:Frage8}) richtet sich wieder an ein korpuslinguistisches Thema. Dabei geht es um die Erfassung von häufig verwendeten Korpussystemen. Wie zu erwarten war, heben sich die drei im deutschen Sprachraum führenden Systeme \emph{TIGERSearch}, \emph{Cosmas II} und \emph{CQP} heraus. Die Häufigkeit von \emph{CQP} wird jedoch dadurch beeinflusst, dass der Fachbereich Korpuslinguistik ein darauf basierendes System mit verschiedenen Korpora betreibt und in die Lehre einbindet. Eine differenzierte Gewichtung der Antworten nach den Nennungsplätzen (1, 2 und 3) ist bei den erfassten Quantitäten nicht möglich.

%Die Nennung von "GOOGLE"\footnote{Google: http://www.google.com} als verwendetes Korpussystem. Auch wenn Google nicht dediziert für sprachwissenschaftliche Analysen entworfen wurde, nutzen einige Linguisten es zur Belegsuche. 

	\begin{table}[H]
		\centering
		\begin{tabular}{l | r r r r}
			 		Frage 8 & 1 & 2 & 3 & \bf{$\sum$}\\
					\hline\\
					
					TIGERSearch 	& 1 & 2 & 1 & \bf{4} \\
					Falko		& 2 & 0 & 0 & \bf{2} \\
					Exmaralda	& 1 & 0 & 0 & \bf{1} \\
					COSMAS/II	& 3 & 1 & 0 & \bf{4} \\
					BNC			& 0 & 1 & 0 & \bf{1} \\
					Annis		& 1 & 0 & 0 & \bf{1} \\
					CELEX		& 0 & 1 & 1 & \bf{2} \\
					GOOGLE		& 0 & 0 & 1 & \bf{1} \\
					Stuttgart Radio 
					News Corpus	& 1 & 0 & 0 & \bf{1} \\
					Kiel Corpora of 
					read and 
					spontaneous 
					speech		& 0 & 1 & 0 & \bf{1} \\
					DWDS		& 1 & 0 & 1 & \bf{2} \\
					CQP			& 5 & 0 & 0 & \bf{5} \\
					SQL basierte 
					lokale DB		& 0 & 1 & 0 & \bf{1} \\
					WordSmith	& 1 & 0 & 0 & \bf{1} \\
					ParaConc		& 0 & 2 & 0 & \bf{1} \\
					TreeTagger	& 0 & 0 & 3 & \bf{1} \\
					The Penn 
					Treebank		& 1 & 0 & 0 & \bf{1} \\
					IDS			& 0 & 1 & 0 & \bf{1} \\
					Sketch Engine	& 0 & 0 & 1 & \bf{1}			
		\end{tabular}
		\caption{Nennen Sie drei Korpussysteme, mit denen Sie arbeiten. Beginnen Sie mit dem wichtigsten.}\label{tab:Frage8}
	\end{table}

In Anschluss an Frage 8 sollten die Befragten eine Selbsteinschätzung zur Beherrschung der  zuvor genannten Korpussysteme abgeben (Tabelle \ref{tab:Frage9}). Hier zeigt sich besonders für die Angaben zum System 1 der von \cite[][S. 180 f.]{korman1971industrial} beschriebene Urteilsfehler der \emph{Zentralen Tendenz} (auch \emph{Tendenz zur Mitte}). So geben 70\% der Befragen einen Kenntnisstand mit dem von ihnen am häufigsten eingesetzten System als ''ausreichend'' oder ''umfangreich'' an. Für die Systeme 2 und 3 haben nur jeweils 45\% bzw. 30\% der Befragen Angaben gemacht. Diese ordnen sich aber deutlich in den Bereich ''ausreichend'' ein.

Eine mögliche Schlussfolgerung aus diesen Angaben ist, dass sich die Arbeit der einzelnen Befragten weitestgehend auf ein System beschränkt und Expertenwissen zu den einzelnen Systemen für linguistische Fragestellungen nicht benötigt wird.
		
	\begin{table}[H]
		\centering
		\begin{tabular}{l | r r r r r}
					Frage 9 & keine Antwort & Rudimentär & Ausreichend & Umfangreich & Experte \\
					\hline\\
					System 1 			& 15\%	& 10\%	& 35\%	& 35\%	& 5\% \\
					System 2			& 55\%	& 5\%	& 30\%	& 5\%	& 5\% \\
					System 3 			& 70\%	& 5\%	& 20\%	& 5\%	& 0\% \\
					\bf{$\varnothing$}	& -----	&	\bf{7}\%	& \bf{32\%}	& \bf{15\%	}& \bf{3\%}
		\end{tabular}
		\caption{Schätzen Sie Ihre Fertigkeiten mit den soeben genannten Systemen ein.}\label{tab:Frage9}
	\end{table}

Die folgenden Fragen 10 bis 14 erfassen die Erfahrung der Benutzer zur Annotation von Korpusdaten. Die Gruppe der Befragten teilt sich dabei zu gleichen Teilen (50\% und 45\%) in jene mit und jene ohne praktische Erfahrung bei der Annotation von größeren Textmengen.

	\begin{table}[H]
		\centering
		\begin{tabular}{l | r}
					Frage 10 &  \\
					\hline\\
					keine Antwort	& 5\% \\
					ja 			& 50\% \\
					nein			& 45\%
		\end{tabular}
		\caption{Haben Sie schon einmal eine größere Textmenge annotiert?} \label{tab:Frage10}
	\end{table}

Gemäß den Angaben zu Frage 11 (Tabelle \ref{tab:Frage11}) sind dabei die Textsorten gleichmäßig vertreten. Nur ''Dialektdaten'' (5\%) und ''Fragebögen'' (0\%) bilden nicht signifikante Ausnahmen. Der hohe Wert für ''Lernerkorpora'' liegt daran, dass an der HU das Lernerkorpus \emph{FALKO} \citep[vgl.][]{luedeling2005lernercorpora} einen Arbeitsschwerpunkt bildet.

	\begin{table}[H]
		\centering
		\begin{tabular}{l | r}
					Frage 11 &  \\
					\hline\\
					Zeitungsartikel			& \bf{20\%} \\
					Gesprochene Sprache	&  15\% \\
					Historische Texte		&  15\% \\
					Lernertexte			& \bf{25\%} \\
					Fragebögen			& 0\% \\
					Dialektdaten			& 5\% \\
					Parallelkorpora 		& \bf{20\%} \\
					Webkorpora 			& 15\% \\
					\hline
					Forendaten 			& 5\% \\
					Literatur (Belletristik)	& 5\% \\
					Nachrichtentexte		& 5\%
		\end{tabular}
		\caption{Welche Textsorten haben Sie bearbeitet?}\label{tab:Frage11}
	\end{table}

Die Fragen 12 und 13 erfassen, welche Annotationen manuell (Tabelle \ref{tab:Frage12}) und welche automatisiert (Tabelle \ref{tab:Frage13}) vorgenommen wurden. Auch hier zeigt sich, dass alle vorgegebenen Kategorien vertreten sind. Bedingt durch deren Häufigkeit in modernen Korpora sind die Annotationsarten für ''Wortarten (POS-Tagging)'' und ''Syntax (Parsing)'' in den Antworten am häufigsten vertreten.

	\begin{table}[H]
		\centering
		\begin{tabular}{l | r}
					Frage 12 &  \\
					\hline\\
					Wortart (POS-Tagging)					& \bf{30\%} \\
					Lemmatisierung						&  15\% \\
					Morphologische Annotationen				&  20\% \\
					Syntax (Parsing)						& \bf{25\%} \\
					Informationsstruktur						& 20\% \\
					Dialektdaten							& 10\% \\
					Semantische Annotation (Sense Tagging) 	& 10\% \\
					Koreferenz 							& 15\% \\
					\hline
					Wortbildung, Clause-Arten 		& 5\% \\
					Phonematik, Orthographie, Prosodie, Textstruktur & 5\% \\
					Alignment					& 5\% 
		\end{tabular}
		\caption{Was für Annotationen haben Sie manuell vorgenommen?}\label{tab:Frage12}
	\end{table}

Die Angaben zu den automatisierten Annotationsarten spiegeln die Realität verfügbarer Werkzeuge wider. Deswegen werden vorrangig ''Wortarten (POS-Tagging)'' und ''Lemmatisierung'' automatisiert vorgenommen \citep[vgl.][]{schmid94treetagger}. Andere Annotationsarten können durch automatische Prozesse oft nur unterstützt werden.
	
	\begin{table}[H]
		\centering
		\begin{tabular}{l | r}
					Frage 13 &  \\
					\hline\\
					Wortart (POS-Tagging)					& \bf{30\%} \\
					Lemmatisierung						&  \bf{30\%} \\
					Morphologische Annotationen				&  10\% \\
					Syntax (Parsing)						& 10\% \\
					Informationsstruktur						& 0\% \\
					Dialektdaten							& 10\% \\
					Semantische Annotation (Sense Tagging) 	& 0\% \\
					Koreferenz 							& 15\% \\
					\hline
					Wortbildung, Clause-Arten 		& 5\% \\
					Phonematik, Prosodie (experimentell) & 5\% \\
					Alignment					& 5\% 
		\end{tabular}
		\caption{Was für Annotationen haben Sie automatisiert vorgenommen?}\label{tab:Frage13}
	\end{table}

Interessant sind die Ergebnisse der 14. Frage (Tabelle \ref{tab:Frage14}) danach, ob die annotierten Daten in einem Korpus-Interface durchsuchbar sind. Hier haben 65\% der Befragten ''keine Antwort'' angegeben. Es ist nicht möglich, auf die Gründe dafür zu schließen. Möglicherweise hat jedoch die Bezeichnung ''Korpus-Interface'' diese Tendenz verursacht, da sie nicht geläufig zu sein scheint.

	\begin{table}[H]
		\centering
		\begin{tabular}{l | r}
					Frage 14 &  \\
					\hline\\
					keine Antwort	& 65\% \\
					ja 			& 20\% \\
					nein			& 15\%
		\end{tabular}
		\caption{Sind diese Daten in einem Korpus-Interface verfügbar? Es kann sich auch um ein nicht öffentliches Korpus handeln.} \label{tab:Frage14}
	\end{table}
	
Um die informationstechnischen Fertigkeiten der Befragten zu erfassen, wurden in Frage 15 (Tabelle \ref{tab:Frage15}) Programmierkenntnisse untersucht. Hier geben immerhin 40\% an, sich mit einer der Skriptsprachen ''Perl / Python / Ruby'' auszukennen. 30\% geben an, auch Erfahrungen mit ''C(++) oder Java'' gemacht zu haben. Schlussendlich kreuzten 40\% keine der Angebotenen Antworten an. Bei denen kann davon ausgegangen werden, dass sie keinerlei Fertigkeiten mit der Programmierung haben.
	
	\begin{table}[H]
		\centering
		\begin{tabular}{l | r}
					Frage 15 &  \\
					\hline\\
					SQL / MS Access	& 20\% \\
					C(++) oder Java  	& 30\% \\
					Perl / Python / Ruby	& 40\% \\
					PHP / ASP		& 15\% \\
					Prolog oder Lisp	& 10\% \\
					keine Nennung	& 40\%
		\end{tabular}
		\caption{Mit welchen der folgenden Programmiersprachen oder Datenbanksystemen haben Sie tiefergehende Erfahrungen.} \label{tab:Frage15}
	\end{table}
	
	
Die Fragen 16 (Tabelle \ref{tab:Frage16}) und 17 (Tabelle \ref{tab:Frage17}) sind technische Anwendungsfragen zur Korpussuche. In Frage 16 wird nach der Berechnung von \emph{regulären Ausdrücken} gefragt. Hier zeigt sich, dass nur 6 Teilnehmer (29\%) alle Teilfragen beantwortet haben. Die Antwortquote liegt für diese Befragten jedoch bei 100\% richtigen Antworten. Bei den restlichen Teilnehmern lag die Quote der korrekten Antworten bei 7/8 (88\%). Diese Ergebnisse können als verlässliches Maß angesehen werden, da \emph{reguläre Ausdrücke} in den meisten Anfragesprachen für Korpora (u.a. CQP) angeboten werden.

	\begin{table}[H]
		\centering
		\begin{tabular}{l | r r r r}
					Frage 16 & Lösung & Antworten & richtig & falsch \\
					\hline\\
					ok	& 3 & 11 & 11 & 0 \\
					.\{12\}  & 1 oder 2  & 7 & 6  & 1 \\
					\^{}.\{12\}\$ & 0 & 7 & 7 & 0 \\
					ok+ & 3 oder 4 & 8 & 6 & 2
		\end{tabular}
		\caption{Geben Sie wenn möglich an, wie viele Teilstrings die folgenden regulären Ausdrücke im Wort ''Rokkokokomode'' finden.} \label{tab:Frage16}
	\end{table}
	
\newpage
Die Frage 17 verlangte die Nennung der Anzahl der Wortformen in folgendem Text.

\begin{verbatim}
Der Meister sagt: "Mit einem horizontal arbeitenden ein-ebenen 
Personenfahrstuhl kann man z.B. nicht auf- und abfahren." 
(s. Abb. 13).
\end{verbatim}
	
Die Antworten auf diese Frage zeigen, dass sie für die Teilnehmer nicht adäquat lösbar war. Dies lässt sich maßgeblich auf die ambige Bedeutung des Begriffs ''Wortform'' zurückführen. So kann z.B. jedes bedeutungstragende Wort (\emph{Lexem}) als ''Wortform'' verstanden werden. Eine andere Interpretation wäre die rein syntaktische Zerlegung des Textes in dessen elementare Einheiten (\emph{Token}). Darüber hinaus erschweren die Mehrdeutigkeiten der Abkürzungen, der Bindestrich-Komposita, der wörtlichen Rede sowie der geklammerten Ausdrücke in dem Text eine konsistente Deutung zwischen den Teilnehmern. Deshalb wird in Tabelle \ref{tab:Frage17} nur die Antwortverteilung aufgeschlüsselt. Weitere Interpretationen sind nicht möglich.

	\begin{table}[H]
		\centering
		\begin{tabular}{l | r r}
					Frage 17 & Antwort & Häufigkeit \\
					\hline\\
					& 9 & 1 \\
					& 11 & 1\\
					& 12 & 1 \\
					& 14 & 2 \\
					& 15 & 2 \\
					& 17 & 2  \\
					& 18 & 1  \\
					& 19 & 2 \\
					& 21 & 1 \\
					& 26 & 1
		\end{tabular}
		\caption{Aus wievielen Wortformen besteht der folgende Text? } \label{tab:Frage17}
	\end{table}	
	
	
Auch wenn die Antworten auf Frage 17 nicht weiter verwertet werden können, so zeigt sich insbesondere durch das Antwortverhalten aus Frage 16, dass die Teilnehmer die eigene Kompetenz im Umgang mit der Korpussuche recht gut einschätzen können. Dies wird auch dadurch gestützt, dass 5 von 6 Teilnehmern, die alle Teilaufgaben aus Frage 16 beantwortet haben, die Kompetenz bei den Fragen 8 und 9 für das angegebene Erstsystem mit ''Ausreichend'' oder ''Experte'' eingeschätzt haben.

Um zu erfahren, auf welchem Wege die Benutzer bisher den Umgang mit Korpussuchen erlernt haben, sollten diese in Frage 18 (Tabelle \ref{tab:Frage18}) die bisherigen Bedeutungen unterschiedlicher Informationsquellen in einer Rangreihe anordnen. Aus den hier gemachten Angaben lässt sich schließen, dass die Mehrheit der Benutzer neue Systeme im Selbststudium erlernt hat. Hierbei wurde besonders auf die verfügbaren Dokumentationen, gefolgt von Tutorials und Hilfefunktionen, zurückgegriffen.

	\begin{table}[H]
		\centering
		\begin{tabular}{l | r r r r r | r}
					Frage 18 & 1 & 2 & 3 & 4 & 5 () & \bf{$\varnothing$}	\\
					\hline\\
					Einführungskurs / Workshop 	& 40\% &   5\%        & 5\%             & 15\%       & 10\% & \bf{1.75} \\
					Online-Tutorial 
					(zielorientierte Hilfe) 		& 10\%        & 15\%      & 20\%          & 10\%         & 20\% & \bf{2.40} \\
					Online-Hilfe 
					(funktionsorientierte Hilfe) 	& 10\%        & 20\%      & 20\%         & 10\%         & 10\% & \bf{2.00} \\
					Dokumentation 			& 10\%        & 20\%        & 25\% & 20\%& 10\% & \bf{2.75} \\
					Persönliche Beratung 		& 25\%        & 25\% & 10\%        & 10\%        & 10\% & \bf{1.95} 
		\end{tabular}
		\caption{Ordnen Sie die folgenden Informationsquellen danach, wie wichtig diese für Ihren Einstieg in die Korpusauswertung waren.}\label{tab:Frage18}
	\end{table}

Unterstütztes Lernen, wie es bei Einführungskursen oder bei persönlicher Beratung stattfindet, scheint eine etwas untergeordnete Rolle einzunehmen. Als Grund dafür kann ein unzureichendes Angebot an solchen Veranstaltungen vermutet werden. Diese Hypothese wird von den Ergebnissen der folgenden Frage 19 (Tabelle \ref{tab:Frage19}) erhärtet. Hier zeigt sich mit einer durchschnittlichen Wertung von 4.05 ein deutlicher Wunsch danach, dass neue Korpussuchen auch Kurse für die Benutzer anbieten. Dennoch soll hier nicht auf die Erstellung von sonstigen Tutorials und Hilfefunktionen sowie auf eine angemessene Dokumentation verzichtet werden. 
	
	\begin{table}[H]
		\centering
		\begin{tabular}{l | r r r r l | r}
					Frage 19 & (eher nicht) 1 & 2 & 3 & 4 & 5 (sehr gern) & \bf{$\varnothing$} \\
					\hline\\
					Kurs 						& 5\% 	 &   10\%        & 5\%            & 30\%       & 45\% & \bf{4.05} \\
					Online- Tutorial oder Hilfe  	& 15\%        & 10\%      & 0\%          	& 30\%         & 25\% & \bf{3.50}\\
					Dokumentation 			& 5\%        & 20\%        & 10\% 		& 35\% & 15\% & \bf{3.41}
		\end{tabular}
		\caption{Wie würden Sie die Benutzung eines neuen Systems am liebsten erlernen?}\label{tab:Frage19}
	\end{table}
	
Die Frage 20 ist die einzige Frage mit der Eingabe eines längeren Freitextes. Sie hat folgenden Wortlaut:

\begin{quote}
Angenommen Sie haben einen Wunsch für Ihr ''perfektes'' Korpussystem frei. Was würden Sie sich wünschen?
\end{quote}

Insgesamt wurde diese Frage von 7 (33\%) Teilnehmern ausgefüllt. Die Kernpunkte der Antworten werden nachfolgend in Stichpunkten aufgeführt.

\begin{itemize}
	\item{intuitive Bedienbarkeit / benutzerfreundliches Interface / klare Anfragesprache}
	\item{ausdruckbare Übersichten über Suchfunktionen}
	\item{umfangreiche Ausgabemöglichkeiten in verschiedenen Formaten}
	\item{Abspeicherung von Suchanfragen}
	\item{Angebot an häufigen Mustersuchen}
	\item{adaptierbare Tutorial-Lösungsangebote für Problem-Szenarien}
	\item{statistische Auswertung (z.B. Verteilungen von Suchergebnissen) mit graphischer Darstellung}
	\item{Kommandozeilen-Interface als alternative Benutzerschnittstelle}
	\item{Möglichkeit der Darstellung prosodischer\footnote{Prosodie beschreibt die hörbaren Eigenschaften gesprochener Sprache. Dazu zählen u.a. der Verlauf von Intonation, Tonhöhe, Lautstärke und Sprechrhythmus.} Annoationen}
	\item{Übernahme von Annotationseben von einem Korpus in ein anderes}
\end{itemize}

Diese Antworten sind insgesamt sehr aufschlussreich. So lassen sich Anforderungen nach umfangreichen Hilfestellungen und der Unterstützung bei der Eingabe von Suchanfragen ableiten. Andere Teilnehmer äußern den Wunsch danach, Anfragen speichern und zu einem späteren Zeitpunkt wieder abrufen zu können. Darüber hinaus zeigt sich jedoch sehr deutlich, wie unterschiedlich die Betrachtungsweisen der Teilnehmer sind. Wo die einen eine sehr technische Sicht auf das System haben (Kommandozeilen-Interface), sind andere Teilnehmer primär an den linguistischen Fragestellungen, die sich mit einem Korpussuchsystem bearbeiten lassen sollten, (prosodische Annotationen) interessiert.

Die letzte Frage 21 (Tabelle \ref{tab:Frage21}) gibt Aufschluss darüber, wie die Verteilung der akademischen Abschlüsse unter den Befragen ist. Hier bilden Magister/Diplomanden mit insgesamt 45\% die Mehrheit, gefolgt von Studenten (kein akademischer Grad) mit 25\% und schießlich 35\% Habilitierte und Doktoren.
	
	\begin{table}[H]
		\centering
		\begin{tabular}{l | r}
					Frage 21 &  \\
					\hline\\
					keinen		& 25\% \\
					Bachelor		&  0\% \\
					Master		&  5\% \\
					Diplom		& 15\% \\
					Magister		& 30\% \\
					Promoviert	& 10\% \\
					Habilitiert 		& 15\%
		\end{tabular}
		\caption{Welchen akademischen Grad haben Sie?}\label{tab:Frage21}
	\end{table}

Nachdem die Ergebnisse der Nutzerbefragung ausführlich besprochen wurden, stehen diese nun als Grundlage für den Entwurf der Benutzerschnittstelle von \emph{ANNIS 2.0} zur Verfügung. Als direktes Entwurfswerkzeug sind sie jedoch ungeeignet. Deshalb werden auf dieser Basis und unter Einbeziehung der Ergebnisse der durchgeführten Interviews im folgenden Abschnitt fiktive Personen zu Beschreibung der Benutzergruppen erarbeitet.

\subsubsection{Die Personas für den Entwurf von \emph{ANNIS 2.0}}\label{sec:Personas}

Die Verwendung von Personas ist eine Methode für den Entwurf von interaktiven Benutzerschnittstellen. Personas gehen auf die Arbeiten von \cite{cooper1999inmates} zum zielorientierten Softwareentwurf zurück und sind heute ein fester Bestandteil des Usability Engineerings. 

Eine Persona ist eine ''precise description of our user and what he wishes to accomplish'' \citep[S. 123]{cooper1999inmates}. \cite{calde2002} beschreiben Personas etwas ausführlicher als ''fictional, detailed archetypical characters that represent distinct groupings of behaviors, goals and motivations observed and identified during the research phase''. Personas können also als fiktive Beschreibung von Benutzern und deren hypothetischen Eigenschaften und Zielen verstanden werden. Die Charaktere bekommen Namen und werden mit einer Hintergrundgeschichte ausgeschmückt. So können diese im Laufe eines Projektes die Orientierungsfunktion real existierender Personen bekommen. Personas werden nach ihren Zielen und Bedürfnissen entworfen. Dazu gehören sowohl persönliche Ziele als auch Aufgaben, die mit dem zu entwerfenden System erreicht werden sollen. 

Wie aus Abschnitt \ref{sec:Nutzerbefragung.Auswertung} hervorgeht, kann die durchgeführte Umfrage allein keine ausreichende Grundlage für die Erstellung der benötigten Personas sein. Es ist an dieser Stelle notwendig, auch auf das in Gesprächen und freien Interviews erlangte Expertenwissen zurückzugreifen. Aus dem daraus entstehenden Gesamtbild der zu erwartenden Benutzergruppen ergeben sich die drei grundlegenden Personas, die im Folgenden vorgestellt werden.\footnote{Die Namen der in dieser Arbeit vorgestellten Personas sind frei erfunden. Die Charakteristika sind Ergebnis der subjektiven Auswertung der Umfrage und anderer Befragungen. Die Identität mit realen Personen wäre zufällig und unbeabsichtigt.} Diese sind in einem subjektiven Prozess entstanden, könnten aber mit den Ergebnissen eines umfangreicheren Fragebogen und einer größeren Teilnehmerzahl mit Hilfe der \emph{Principal Components Analysis PCA} \citep[ein frühes Beispiel geben][]{jolicoeur1960pca} statistisch untermauert werden.

\newpage
\begin{description}
	
	\item[Sophie] {
		studiert im sechsten Semester Germanistik. Sie möchte sich schon auf ihre bevorstehende Abschlussarbeit vorbereiten und besucht deshalb einen Einführungskurs der Korpuslinguistik.
		
		Hier lernt sie neben den theoretischen Grundlagen auch den Umgang mit verschiedenen Korpussystemen. Die Anfragesprachen bereiten Ihr besondere Probleme. Gerade die unterschiedlichen Semantiken der verschiedenen Anfragesprachen und Funktionalitäten der einzelnen Systeme führen immer wieder zu Verwechslungen. Sie kennt sich in der Programmierung nicht aus und auch reguläre Ausdrücke bereiten ihr Schwierigkeiten.
		
		Sie wünscht sich für jedes Korpussystem eine knappe Zusammenfassung mit Beispielen auf einer Seite. Ausführliche Dokumentationen findet sie eher abschreckend und benutzt diese nur wenn es für die Problemlösung unbedingt notwendig ist. Am liebsten wäre es Sophie aber, wenn sie alle ihre Suchen über verschiedene Korpora nur ein einziges System beherrschen müsste.
	}
	
	\item[Konrad] { ist Korpuslinguist. Sein Interesse gilt seit mehr als drei Jahren der Erforschung von Sprachphänomenen mit Hilfe von Korpora. Er beherrscht die statistische Auswertung von Korpussuchen und erachtet diese als notwendiges Mittel für korrektes wissenschaftliches Arbeiten.
		
		Er hat schon mit vielen verschiedenen Korpussystemen gearbeitet und auch Annotationen von Korpusdaten erstellt. Konrad hat grundlegende Programmierkenntnisse und kennt sich mit Datenbanken aus. Deswegen baut und pflegt er einige kleine Werkzeuge für seine eigenen Arbeiten. Er hätte auch gerne eine Kommandozeile für das Korpussystem.
		
		Für die Arbeit mit einem ihm unbekannten System braucht er lediglich eine Kurzbeschreibung der Anfragesprache und Informationen über die Korpora. Er ist aber auch in der Lage, vollständige Dokumentationen zu benutzen, um sich mit den Spezialfunktionen der Systeme vertraut zu machen. Dabei interessieren ihn auch Details zu den Annotationsschemata und deren Besonderheiten. Auf dieser Grundlage kann er dann alle erwünschten Anfragen formulieren und sogar seinen Kollegen mit Rat und Tat zur Seite stehen.
	}

	\item[Lisa] {
		ist schon seit mehreren Jahren wissenschaftlich in der Linguistik tätig. Sie hat Expertenwissen in dem eigenen Forschungsgebiet und ist im Umgang mit Software für entsprechende Zwecke erfahren. Die Benutzung von Korpora für wissenschaftliche Veröffentlichung spielte für sie aus verschiedenen Gründen bisher keine oder nur eine untergeordnete Rolle. Statistische Berechnungen über Ergebnissen von Korpussuchen hat Sie noch nicht durchgeführt.
		
		Lisa verspricht sich von der Korpussuche eine schnelle und flexible Zugriffsmöglichkeit auf die von ihr erstellten Korpora und auch auf andere interessante Korpora. Sie erwartet, dass die Suchschnittstelle intuitiv benutzbar ist und legt großen Wert auf Tutorials mit Beispielen in allen Schwierigkeitsgraden, um daraus ihre eigenen Anfragen ableiten zu können.
		
		Der Import der von Lisa erstellten Daten ist oft sehr aufwendig, da diese von Menschen manuell (z.B. mit \emph{Microsoft Excel}\footnote{Microsoft Excel: http://office.microsoft.com/excel}) erstellt wurden und für die Weiterverarbeitung mit angepassten Programmen konvertiert werden müssen. Sie ist aber bereit, für einen schnelleren Import und die Vorteile der Korpussuche zukünftige Korpora in Absprache mit Korpuslinguisten zu erstellen.
		
		
		Die Erstellung ihrer Korpora gestaltet sich oft als iterativer Prozess in dem die annotierten Daten in ein Korpussystem eingespielt, dort auf Konsistenz untersucht und danach wieder in den Annotationswerkzeugen angepasst werden. Dieser aufwendige Prozess benötigt viel Ressourcen und sie möchte deshalb genau bestimmen können, wer auf ihre Daten zugreifen kann.
	}
\end{description}

Diese fiktiven Personas vermitteln ein intuitives Verständnis von den erwarteten Benutzern und ergänzen die grundlegenden Anforderungen, wie sie aus den Normen (Abschnitt \ref{sec:Normen}), den Usability Heuristiken von Nielsen (Abschnitt \ref{sec:Heuristiken}) und der Benutzerbefragung (Abschnitt \ref{sec:Nutzerbefragung}) hervorgehen. Bevor diese aber abschliessend zusammengefasst werden können, muss eine wichtige technische Überlegung hinzugefügt werden.

\newpage
\subsection{Web vs. Desktop}\label{sec:Desktopanwendung}

Interaktive Web-Applikationen unterscheiden sich in ihren Interaktionsparadigmen stark von grafischen Benutzeroberflächen, wie man sie von lokal installierten Anwendungsprogrammen kennt. Die Benutzung einer Web-Applikation unterliegt prinzipiell dem Paradigma einzelner Seiten, zwischen denen man im Laufe der Benutzung wechselt. Eingaben werden über in die Seiten eingebettete Formulare oder einzelne Formularfelder vorgenommen. Damit eine Nutzereingabe vom Web-Server verarbeitet werden kann, muss diese per Klick auf einen Knopf bestätigt werden. Das Ergebnis der Verarbeitung wird dann auf einer neuen Seite dargestellt. Abbildung \ref{fig:WebApplication.Informationsfluss} illustriert den Ablauf dieses Vorgangs, in dem der Benutzer sich stets dem Wechsel zwischen einzelnen Seiten befindet, am Beispiel des Abschlusses eines Kaufs in einem Online-Shop. Der Prozess beginnt mit der Ansicht des Warenkorbes und geht dann über die Eingabe der Kreditkarteninformationen zur Anzeige der Kaufbestätigung.

Dieser Seiten- und damit Kontextwechsel impliziert zusätzliche kognitive Last im Arbeitsgedächtnis des Benutzers, da sich dieser oft den Gesamtzustand der Webanwendung vorhalten muss. \cite{nielsen1997metaphor} vertritt daher die Meinung, dass die Metapher \emph{Telefon} für das Web am zutreffendsten ist. 

\begin{figure}[H]
	\centering
	\includegraphics*[width=0.5\textwidth]{figures/DA/WebApplication_Informationsfluss.pdf}
	\caption{Informationsfluss in einer klassischen Web-Applikation}\label{fig:WebApplication.Informationsfluss}
\end{figure}

Einige dieser Nachteile lassen sich durch den Einsatz von \emph{JavaScript} und optimierten Seitenaufteilungen gut ausgleichen. Dadurch können einfache Anwendungen wie z.B. Webshops oder Datenbankdienste effizient realisiert werden. Da bei \emph{ANNIS} die tatsächlichen Anwendungsgebiete und Nutzungsmotivationen nicht genau vorhergesagt werden können, muss hier auf das weitaus flexiblere Gestaltungsparadigma des \emph{Graphical User Interface GUI} zurückgegriffen werden. Im Folgenden wird der Begriff \emph{GUI} im Sinne des von Merzouga Wilberts 1980 vorgestellten \emph{WIMP}-Paradigmas verstanden. Das bedeutet, dass ein \emph{GUI} die Kombination von \emph{Windows} (Fenster), \emph{Icons} (klickbare Symbole), \emph{Menus} (Menüs) und einem \emph{Pointing Device} (z.B. einer Maus oder einem Touchpad) sind. Der Begriff \emph{WIMP} wird oft auch als Synonym für \emph{GUI} gebraucht. Es bleibt anzumerken, dass \cite{dam97postwimp} dieses Konzept wegen neuer Eingabegeräte wie z.B. Handschrift- und Gestenerkennung auf Handheld-Geräten kritisiert und Erweiterungen vorschlägt. Davon ist jedoch die Anwendung im Rahmen dieser Arbeit nicht betroffen.

\emph{GUI}s sind eine weit verbreitete Form der Mensch-Computer-Interaktion, die jedem Benutzer eines Computers bekannt sein sollte. Tabelle \ref
{tab:WEBvsGUI} stellt noch einmal wesentliche Eigenschaften von Web-Applikationen und \emph{GUI}s gegenüber.


\begin{table}[H]
		\begin{tabular}{l | p{165pt} | p{165pt}}
					& \bf{\emph{GUI} (WIMP)} 	& \bf{traditionelle Web-Applikation}	\\
					\hline\\
			\bf{Darstellung} & Der Entwickler kann die Darstellung pixelgenau kontrollieren. & Gerätevielfalt sorgt für unterschiedliche Darstellungen. \\
			\bf{Kontrolle} 	& Der Entwickler kann den Dialogfluss genau kontrollieren.	&  Der Benutzer kontrolliert den Fluss über das Ansteuern von Seiten. \\
			\bf{Kontext}	& Der Kontext wird allein über die Darstellung realisiert. Der Benutzer kann den Status der Anwendung schnell erfassen. & Seitenkonzept impliziert einen zusätzlichen Kontext, den der Benutzer mit berücksichtigen muss. \\
			\bf{Einbettung}	& Die Anwendung ist immer in die Oberfläche des Betriebssystems eingebettet und folgt dessen Darstellungsregeln. & Die Web-Applikation wird im Kontext des Web-Browsers ausgeführt. Die Darstellungsparadigmen des Betriebssystems gelten nicht. Der Applikationscharakter wird verschleiert. \\
			\bf{Zustand} & Der Zustand der Anwendung ist nur temporär und kann i.A. nicht ohne Umwege wieder\-hergestellt werden. & Da die Systemzustände über URLs repräsentiert werden, können diese i.A. über Lesezeichen wiederhergestellt werden. Dies gilt nicht, wenn Systemzustände sitzungsbasiert verwaltet werden.\\
		\end{tabular}
		\caption{Gegenüberstellung wichtiger Unterschiede zwischen traditionellen Web-Applikationen und fensterorientierten \emph{GUI}s \citep[vgl.][]{nielsen1997difference}}\label{tab:WEBvsGUI}	
\end{table}

\cite{nielsen1997difference} betont insbesondere den Unterschied zwischen den Geräten, auf denen die Anwendung dargestellt wird. Wo für \emph{GUI}s meist Gestaltungsrichtlinien vom Hersteller des Betriebssystems berücksichtig werden und die Anzeige pixelgenau kontrollierbar ist, werden Webseiten auf vielen unterschiedlichen Geräten angezeigt. Die Benutzer verwenden sowohl High-End-Personalcomputer als auch stiftbasierte Handheld-Geräte. Man kann also davon ausgehen, dass sich die Darstellungen im Detail sehr stark unterscheiden können.

Spezialisierte Web-Applikationen wie \emph{ANNIS 2.0} haben hier einen weitaus engeren Rahmen. Die Benutzergruppe und der Anwendungsrahmen sind bekannt. Ebenso gewinnen Benutzer an Erfahrung und Routine in der Benutzung. Daraus lassen sich Folgerungen über die Endgeräte und damit den nötigen Aufwand für eine einheitliche Darstellung ziehen. Es stehen ausschließlich die Funktionen und Features im Vordergrund, die zur effektiven und effizienten Bearbeitung der Problemstellungen notwendig sind. Nachfolgend werden die Anforderungen an \emph{ANNIS 2.0} abschließend zusammengefasst.

\subsection{Anforderungen an \emph{ANNIS 2.0}}\label{sec:Anforderungen an ANNIS 2.0}

Aus den bisher gemachten Vorüberlegungen  lassen sich folgende wesentliche Anforderungen an \emph{ANNIS 2.0} ableiten. 

\begin{description}
	\item[Intuitive Bedienung]{Die Suche mit dem System soll so intuitiv wie möglich gestaltet sein. Die Funktionalität soll auf das Notwendigste reduziert sein. Alle für die Suche notwendigen Einstellungen müssen ohne Umwege einstellbar sein.}
	\item[Sofortige Rückmeldung]{Direkt nach der Eingabe der Suchanfrage soll Rückmeldung darüber gegeben werden, wie viele Fundstellen es in den selegierten Korpora gibt. Syntaktische und semantische Fehler im Suchausdruck sollen möglichst detailliert gemeldet werden.}	\item[Zielorientierte Hilfe]{Es sollen Tutorials für viele Arten von Anfragen angeboten werden, aus denen sich die Benutzer neue Anfragen ableiten können.}
	\item[Vielfältige Darstellungsmöglichkeiten]{Annotationsebenen müssen anhängig von deren Inhalt und Semantik jeweils auf die bestmögliche Weise dargestellt werden.}
	\item[Funktionsorientierte Hilfe]{Es sollen Hilfstexte sowohl zu allen Funktionen als auch zur Anfragesprache \emph{AQL} (siehe Abschnitt \ref{sec:AQL}) leicht abrufbar sein.}
	\item[Exportierbarkeit]{Es ist notwendig, dass die gefundenen Ergebnisse zur Weiterverarbeitung und Veröffentlichung aus dem System exportiert werden können. Hier sollen unterschiedliche Formate angeboten werden.}
\end{description}

Neben diesen Anforderungen zur Benutzbarkeit der Software müssen auch technische Faktoren berücksichtigt werden.

\begin{description}
	\item[Implementierung als Web-Anwendung]{Die Implementierung von \emph{ANNIS 2.0} als Web-Anwendung erspart dem Benutzer das nicht selten aufwendige lokale Installieren von Software und vereinfacht das Ausrollen von Updates und Bugfixes. Die Abdeckung aller gängigen Web-Browser ist dafür jedoch notwendig.	
	\item[Erweiterbarkeit/Wartbarkeit]{Dass eine Anwendung mit geringem Aufwand erweitert und gewartet werden kann, ist eine verbreitete Annahme. In der Praxis zeigt sich jedoch oft, dass Anwendungen aufgrund unzureichender Dokumentationen und mangelhafter Projektpflege einschlafen. \emph{ANNIS 2.0} soll modular und flexibel implementiert werden. Die Bereitstellung aller Quelltexte als \emph{OpenSource} soll auch anderen wissenschaftlichen Projekten die Nutzung und Erweiterung ermöglichen.}
	\item[Unterstützung]{Für das System muss von der betreibenden Gruppe Unterstützung angeboten werden. Hierbei steht zwar die Unterstützung der Datenaufbereitung und des Imports im Vordergrund, dennoch sollen sich die Benutzer auch mit Fragen zur Suche an das Team wenden können.}
}
\end{description}

Im Rahmen dieser Arbeit wurde eine Benutzerschnittstelle, die diesen Anforderungen genügt, implementiert. Deren Einbettung in das Gesamtsystem sowie die Umsetzung einzelner Teilaspekte werden im nächsten Kapitel beschrieben.

\newpage
\section{Die Umsetzung von \emph{ANNIS 2.0}}\label{sec:ANNIS 2.0}

Dieses Kapitel stellt die im Rahmen dieser Arbeit entwickelte Benutzerschnittstelle von \emph{ANNIS 2.0} vor. Dazu wird diese zunächst in den Kontext des Gesamtsystems eingebettet. Danach werden in Abschnitt \ref{sec:DDDQuery} weiterführende Informationen und Verweise zum zugrunde liegenden Suchsystem \emph{DDDQuery} gegeben. Dann wird ab Abschnitt \ref{sec:Web-Applikation} die Web-Applikation und deren Teilkonzepte zur Benutzerinteraktion sowie zum Datenaustausch zwischen Web-Server und Web-Browser vorgestellt. Schließlich werden die Features \emph{ANNIS.build} (Abschnitt \ref{sec:ANNIS.build}), der Authentifizierungsmechanismus (Abschnitt \ref{sec:Authentifizierung}) sowie ein neuartiges Zitiersystem in Abschnitt \ref{sec:Zitieren} beschrieben.

\subsection{Aufbau des Gesamtsystems}

Die Benutzerschnittstelle \emph{ANNIS 2.0} kann nicht eigenständig existieren. Es wird ein Backend-System benötigt, das die vom Benutzer eingegebenen Suchanfragen verarbeitet und die Ergebnisse mit allen verfügbaren Informationen zur Verfügung stellt. Darüber hinaus greift das Backend-System zur Verwaltung der großen Datenmengen auf ein Datenbanksystem zu. Auf der anderen Seite ist der Web-Browser der Einstiegspunkt für den Benutzer. Durch die Verwendung von \emph{JavaScript} kann hier ein bedeutender Teil der Visualisierung ausgeführt und damit der Webserver, auf dem die Web-Applikation läuft, entlastet werden.

\begin{figure}[H]
	\centering
	\includegraphics*[width=0.9\textwidth]{figures/DA/Informationsfluss.pdf}
	\caption{Informationsfluss von der Suchanfrage bis zur Ergebnisliste durch das Korpussystem}\label{fig:Informationsfluss}
\end{figure}

Abbildung \ref{fig:Informationsfluss} fast den Informationsfluss einer einzelnen Suchanfrage durch das gesamte Korpussystem zusammen. Die Anfrage kann entweder manuell in das Suchfeld eingegeben oder mit Hilfe des grafischen Anfragen-Editors \emph{ANNIS.build} (siehe Abschnitt \ref{sec:ANNIS.build}) erstellt worden sein. Die Anfrage wird mit \emph{JavaScript} an den Web-Server weitergeleitet und dort für diese Sitzung gespeichert. Fordert der Benutzer die Ergebnisliste an, so wird die zuvor hinterlegte \emph{ANNIS QL} Anfrage an den \emph{ANNIS}-Service zur Bearbeitung weitergegeben. 

Der Service muss die Anfrage über den \emph{AQL/DDDQuery-Compiler} und anschließend den \emph{DDDQuery/SQL-Compiler} nach \emph{SQL}\footnote{SQL: Die \emph{Structured Query Language SQL} ist eine Anfragesprache für relationale Datenbanken.} überführen. Das \emph{SQL} kann dann direkt vom angeschlossenen Datenbanksystem ausgeführt und das Ergebnis als  \emph{DB Result Set} zurückgegeben werden.

\subsection{Suche mit DDDQuery}\label{sec:DDDQuery}

Der \emph{ANNIS-Service} baut für die Suche direkt auf \emph{DDDQuery} auf. \emph{DDDQuery} ist das von \cite{vitt2005dddquery} erarbeitete Abfragesystem für das von \cite{dipper2004ddd} vorgeschlagene DDD-Korpus. Es ermöglicht die Überführung von beliebigen Korpora in ein relationales Datenbankschema. Diese Datenbanken können dann über eine eigens entwickelte Anfragesprache durchsucht werden. Dafür übersetzt ein Compiler die Anfragen nach \emph{SQL}. In \cite{rosenfeldXXXX} wird die für \emph{ANNIS} modifizierte Version von \emph{DDDQuery} und deren Implementierung als transparenter \emph{Java-RMI}-Service beschrieben.

Die Anfragesprache \emph{DDDQuery} basiert \emph{XPath}, einer Anfragesprache für die Selektion von Knoten aus \emph{XML}-Dokumenten. Obwohl sie dadurch zwar sehr umfangreiche Suchmöglichkeiten für Baumstrukturen bietet, ist die Syntax für die effiziente Eingabe von linguistischen Suchanfragen jedoch gänzlich ungeeignet. Die folgende Gegenüberstellung von \emph{AQL}-Beispielen (\textbf{fett} gedruckt) und den korrespondierenden \emph{DDDQuery}-Anfragen soll als Beleg dafür dienen.

\begin{description}
	\item["Haus"]{element(STRUCT)\#(t0)\$t0/element-span::exact-match(''Haus'')}
	\item[tiger:pos=''ADJA''\ \& ''Haus''\ \& \#1 . \#2]{( ( element(ANNO)\#(a0)[attribute\\ ::attribute(tiger) = 'pos']\$a0/child::element(STRUCT)\#(t0)\$t0/element-span\\ ::exact-match(''ADJA'') )\ \& ( element(STRUCT)\#(t0)\$t0/element-span\\ ::exact-match(''Haus'') ) )\ \& ( \$v1/immediately-following::\$v2 }
	\item[tiger:cat=''NP''\ \& ''Haus''\ \& \#1 >* \#2]{( ( element(ANNO)\#(a0)[attribute\\::attribute(tiger) = 'cat']\$a0/child::element(STRUCT)\#(t0)\$t0/element-span\\::exact-match(''NP'') )\ \& ( element(STRUCT)\#(t0)\$t0/element-span\\::exact-match(''Haus'') ) )\ \& ( \$t2/ancestor::element()\$t1 )}
\end{description}


Desshalb ist \emph{DDDQuery} in \emph{ANNIS 2.0} nur ein Zwischenschritt zwischen der vom Benutzer eingegebenen \emph{AQL}-Anfrage (siehe Abschnitt \ref{sec:AQL}) und der im Datenbanksystem ausgeführten \emph{SQL}-Anfrage. Benutzer werden zu keinem Zeitpunkt mit\emph{DDDQuery} konfrontiert. Alle Ein- und Ausgaben geschehen über die Web-Anwendung \emph{ANNIS 2.0}, die im folgenden Abschnitt erläutert wird.

\subsection{Die \emph{Java} Web-Applikation \emph{ANNIS 2.0}}\label{sec:Web-Applikation}

Wie in Abschnitt \ref{sec:Desktopanwendung} beschrieben, ist \emph{ANNIS 2.0} als Web-Applikation implementiert. Das bedeutet, dass die Benutzer mit einem Web-Browser auf die Anwendung zugreifen. Die Installation von Software ist nicht notwendig.

\begin{figure}[H]
	\centering
	\includegraphics*[width=0.9\textwidth]{figures/DA/WebApplication}
	\caption{Die Web-Applikation \emph{ANNIS 2.0} übernimmt die Berechnung und Darstellung von Ergebnissen.}\label{fig:WebApplication}
\end{figure}

\newpage
\emph{ANNIS 2.0} ist in \emph{JAVA 5.0} implementiert und besteht vorwiegend aus Servlets\footnote{Servlets sind Java Klassen, die das Interface \emph{javax.servlet.Servlet} implementieren. Sie nehmen HTTP Anfragen entgegen und geben das Ergebnis i.A. über HTTP aus.}, die über \emph{Java RMI}\footnote{\emph{Java RMI}: Konzept zur Ausführung von Methodenaufrufen zwischen verschiedenen Java Laufzeitumgebungen (JVM).} mit dem \emph{ANNIS-Service} kommunizieren.
In Abbildung \ref{fig:WebApplication} wird das Zusammenspiel zwischen den einzelnen Servlets der Web-Applikation und dem Web-Browser des Benutzers illustriert. 

Der wichtigste Teilaspekt bei der Umsetzung der Web-Applikation ist die Verarbeitung von Korpusdaten. Diese müssen dafür einheitlich transportiert und möglichst effizient weiterverarbeitet werden können. Diese Voraussetzungen erfüllt das Austauschformat \emph{PAULA-Inline}.

Die einzelnen Fundstellen (\emph{DBResultSet}) werden sequenziell mit allen dazu gehörenden Annotationen im \emph{PAULA-Inline}-Format angereichert und in ein \emph{AnnisResultSet} übertragen. Dieses wird dann an die \emph{ANNIS 2.0} Web-Applikation weitergegeben.

\emph{ANNIS 2.0} speichert das \emph{PAULA} der einzelnen Fundstellen in einem Festplattencache\footnote{Ein \emph{Cache} ist ein Speicher, der kürzlich verwendete Daten für den schnellen Neuzugriff bereit hält.} auf dem Web-Server, um den \emph{ANNIS}-Service und die Netzwerkverbindung zu entlasten. Die \emph{PAULA}-Daten werden bei Bedarf von den Visualisierungsmodulen erneut geladen und weiterverarbeitet.

Die für den Web-Browser notwendigen Daten zur Darstellung der Ergebnisliste werden im \emph{JSON}-Format (siehe Abschnitt \ref{sec:JSON}) übertragen. Die Darstellung erfolgt mittels \emph{JavaScript} und dem \emph{ExtJS} (siehe Abschnitt \ref{sec:ExtJS}) Framework.


\newpage
\subsection{Fensterorientierte Suchoberfläche mit \emph{ExtJS}}\label{sec:ExtJS}

Die Entwicklung einer fensterorientierten Oberfläche für Web-Browser ist eine sehr komplexe Aufgabe. Es müssen sowohl alle Facetten des modernen HTML-Layouts als auch deren Steuerung mit Hilfe von \emph{JavaScript} kombiniert werden. Spätestens aber die Kompatibilität mit allen gängigen Web-Browsern nimmt in einem solchen Projekt viele Ressourcen in Anspruch. Da dies im Rahmen von der Entwicklung von \emph{ANNIS 2.0} nicht umgesetzt werden kann, wird auf das frei verfügbare \emph{ExtJS}\footnote{ExtJS Website: http://extjs.com} Framework zurückgegriffen.

\emph{ExtJS} bietet alle notwendigen Funktionen zur Implementierung von vollständigen \emph{GUI}s für den Web-Browser. Dazu gehören u.a.:

\begin{itemize}
	\item{Fenster}
	\item{Layout-Manager (Spaltenlayouts, Gitterlayouts)}
	\item{Formularfelder (Textboxen, Buttons, Radiobuttons, Checkboxen)}
	\item{Grids zur tabellarischen Darstellung und Bearbeitung von Daten}
	\item{Events}
	\item{Hilfsfunktionen für die asynchrone Kommunikation mit dem Web-Server}
\end{itemize}

Gerade der letzte Punkt ist für die Entwicklung einer interaktiven GUI für das Web unerlässlich. Der nächste Abschnitt geht detaillierter auf diesen Aspekt ein.

\newpage
\subsubsection{Asynchrone Kommunikation mit \emph{AJAX}}\label{sec:AJAX}

\emph{AJAX} steht für \emph{Asynchronous JavaScript and XML} und ist eine technische Methode zum dynamischen Laden von Teilen von Webseiten mittels JavaScript. Das \emph{Web Standards Project}\footnote{\emph{The Web Standards Project} Homepage: http://www.webstandards.com} definiert es als:

\begin{quote}
	(\emph{AJAX} is) A scripting technique for silently loading new data from the server. Although \emph{AJAX} scripts commonly use the soon to be standardized XMLHttpRequest object, they could also use a hidden iframe or frame.
		
	An \emph{AJAX} script is useless by itself. It also requires a DOM\footnote{DOM: Document Object Model, Objektrepräsentation des HTML Seiteninhaltes. Diese kann z.B. mit JavaScript verändert werden.} Scripting component to embed the received data in the document.
\end{quote} 

\begin{figure}[H]
	\centering
	\includegraphics*[width=0.9\textwidth]{figures/DA/AJAX_Methodologien.pdf}
	\caption{Methodologien des Einsatzes von \emph{AJAX}}\label{fig:AJAX.Methodologien}
\end{figure}

Für das dynamische Laden von Seiteninhalten ergeben sich zwei unterschiedliche Methodologien, wie sie in Abbildung \ref{fig:AJAX.Methodologien} gegenüber gestellt werden. In einfacheren Anwendungsfällen bietet sich die Übertragung von HTML-Inhalten an. Diese können statisch auf dem Server vorliegen oder von diesem pro Anfrage dynamisch generiert werden. Diese Nutzungsweise hat den Vorteil, dass auf Empfängerseite keine weiteren Verarbeitungsschritte notwendig sind. Die HTML-Inhalte können direkt in die angezeigte Seite eingebaut werden. Nachteilig ist jedoch, dass Layout-Änderungen oft eine Anpassung dieser server-seitigen Komponenten notwendig machen. Konzeptionell ist auch zu berücksichtigen, dass die Daten nur noch mit erheblichem programmatischem Aufwand im Web-Browser weiterverarbeitet werden können. Die Wiederverwendbarkeit der beteiligten Komponenten wird reduziert.

Aus diesen Überlegungen heraus hat es sich als gängige Praxis etabliert, \emph{AJAX} hauptsächlich für die Kommunikation mit wiederverwendbaren serverseitigen Komponenten zu verwenden. Hierbei werden die Daten vom Web-Browser via HTTP vom Web-Server angefordert, dort berechnet und zum Web-Browser übertragen. Im Web-Browser werden die Daten dann mit \emph{JavaScript} weiterverarbeitet und durch Manipulation des DOM in die Seite integriert. Ein gängiges Austauschformat für diesen Zweck ist \emph{JSON}.

\subsubsection{Das Austauschformat \emph{JSON}}\label{sec:JSON}

Die \emph{JavaScript Object Notation JSON} ist ein sprachunabhängiges textbasiertes Austauschformat für strukturierte Daten. Mit \emph{JSON} übertragene Objekte lassen sich sehr einfach in \emph{JavaScript}-Objekte überführen. Das ist der große Vorteil von \emph{JSON} gegenüber dem Serialisierungsformat XML. In \emph{ANNIS 2.0} wird es für die Übertragung zwischen dem Web-Server und dem Web-Browser von folgenden Informationen verwendet.

\begin{itemize}
	\item{Übertragung der Korpusliste}
	\item{Übertragung der Trefferlisten (verfügbare Annotationsebenen, Tokenannotationen)}
\end{itemize}

\emph{JSON} ist jedoch für die Übertragung von hochstrukturierten linguistischen Annotationsdaten (z.B. Syntaxbäume) nur bedingt geeignet, denn das Parsen und die Weiterverarbeitung dieser Daten mittels \emph{JavaScript} im Web-Browser der Benutzer würde die Komplexität des Systems und die Laufzeiten einzelner Verarbeitungsschritte erhöhen. Aus diesem Grund werden alle komplexen Darstellungen in \emph{ANNIS 2.0} von Servlets direkt aus dem \emph{PAULA-Inline} (siehe Abschnitt \ref{sec:Korpussuche.Datenformat.PAULA-Inline}) generiert und deren Ausgabe unverarbeitet in die finale HTML-Darstellung übernommen.

Die primitiven Datentypen von \emph{JSON} sind \emph{string}, \emph{number} und \emph{boolean}. Sie können in den komplexen Typen \emph{object} und \emph{array} zusammengefasst werden. Der Typ \emph{array}  repräsentiert eine Liste von primitiven oder komplexen Datentypen, \emph{object} repräsentiert eine assoziative Liste von primitiven oder komplexen Typen. Die Schlüsselwerte für die Assoziationen in \emph{object} sind vom primitiven Datentyp \emph{string}. Die vollständige Grammatik von \emph{JSON} befindet sich auf der Homepage http://json.org. 

\begin{figure}[H]
\scriptsize
\begin{verbatim}
{
  {
    "size"	: 3,
    "list"	: [
      {
        "id"			: 7,
        "name"		: "c6.hindi",
        "textCount"		:1,
        "tokenCount"	:2218
      },
      {
        "id" : 5,
        "name" : "b4.muspilli",
        "textCount" : 1,
        "tokenCount" : 909
      },
      {
        "id" : 11,
        "name" : "pcc-11",
        "textCount" : 11,
        "tokenCount" : 1939
      }
    ]
  }
}
\end{verbatim}
\caption{\emph{JSON} Kodierung von Korpusmetadaten}\label{src:JSONExample}
\end{figure}


\begin{table}[H]
	
		\centering
		\begin{tabular}{r | l r r}
					id & Name & Texte & Token\\
					\hline
					7 	&	c6.hindi		& 1	& 2218 	\\
					5	&	b4.muspilli	& 1	& 909	\\
					11	&	pcc-11		& 11	& 1939	
		\end{tabular}
		\caption{Darstellung der in Abbildung \ref{src:JSONExample} kodierten Korpusmetadaten.}\label{tab:JSONExample}
\end{table}


Die Abbildung \ref{src:JSONExample} sowie die Tabelle \ref{tab:JSONExample} zeigen die \emph{JSON}-Kodierung einer Liste von Korpora sowie die Anzahl der darin enthaltenen Token und Texte und eine mögliche Darstellung dieser Informationen.




\subsection{Darstellung von Ergebnissen}\label{sec:Ergebnislisten}

Nachdem nun mit der Einführung von \emph{AJAX} und \emph{JSON} die technischen Grundlagen für die Umsetzung der Benutzerschnittstelle \emph{ANNIS 2.0} gelegt wurden, soll in den folgenden Abschnitten die Darstellung der Ergebnisse ausführlicher besprochen werden.  Dazu werden zunächst die Ergebnislisten vorgestellt. Danach wird näher auf die Darstellungen von Spans in der \emph{Partituransicht} (Abschnitt \ref{sec:Darstellung.Partitur}), von \emph{Syntaxbäumen} (Abschnitt \ref{sec:Darstellung.Syntaxbäume}) und von Pointing Relations in der Koreferenzansicht (Abschnitt \ref{sec:Darstellung.Koreferenz}) eingegangen.

Nachdem eine Suchanfrage erfolgreich ausgeführt wurde, müssen die Ergebnisse in geeigneter Form dargestellt werden. Die wichtigsten Anforderungen an diese Ergebnislisten können wie folgt zusammengefasst und mit den zuvor formulierten Anforderungen aus Abschnitt \ref{sec:Anforderungen an ANNIS 2.0} verknüpft werden.

\begin{enumerate}
	\item{Der Benutzer soll schnell einen Überblick über die gefundenen Stellen bekommen, was aus der Anforderung nach direkter Rückmeldung hervor geht.}
	\item{Die Hauptreferenz einer Fundstelle soll deren Text sein. Dadurch wird dem Benutzer die gewohnte Sicht auf den ursprünglichen Text leicht ermöglicht.}
	\item{Es sollen alle verfügbaren Annotationsebenen einer Fundstelle mit dieser assoziiert angezeigt werden können, was unmittelbar der zu einer Reduktion der kognitiven Last führt und umständliches Umschalten zwischen verschiedenen Ansichten vermeidet.}
\end{enumerate}

\newpage
In Anlehnung an die Arbeiten von \cite{prze2004ipipan} und der daraus entstandenen Korpussuche \emph{Poliqarp} (Abbildung \ref{fig:ErgebnislistePoliqarp}) in den folgenden drei Spalten aligniert:

\begin{description}
	\item[\emph{Left Context}]{bezeichnet den linken Kontext vor der eigentlichen Fundstelle;}
	\item[\emph{Match}]{bezeichnet die eigentliche Fundstelle;}
	\item[\emph{Right Context}]{bezeichnet den rechten Kontext nach der eigentlichen Fundstelle.}
\end{description}

\begin{figure}[H]
	\centering
	\includegraphics*[width=1.0\textwidth]{figures/DA/Poliqarp_3.jpg}
	\caption{Ergebnisliste von \emph{Poliqarp} mit \emph{Left Context}, \emph{Match} und \emph{Right Context}\citep[vgl.][]{prze2004ipipan}}\label{fig:ErgebnislistePoliqarp}
\end{figure}

In der ersten Spalte befinden sich übereinander rechtsbündig angeordnet alle Textanteile, die zum vorangestellten Kontext der Fundstelle gehören. In der mittleren Spalte sind alle eigentlichen Fundstellen zentriert übereinander angeordnet. Die rechte Spalte richtet die der Fundstelle nachgestellten Textanteile linksbündig aneinander aus. Es kann vermutet werden, dass durch diese Anordnung die wesentlichen textuellen Eigenschaften der Fundstellen mit geringem kognitiven Aufwand vom Benutzer visuell erfasst und verglichen werden können. 

Darüber hinaus werden die Ergebnislisten seitenweise präsentiert, was sowohl den Platzbedarf auf dem Bildschirm als auch die Ladezeiten dieser Ansichten stark reduziert. Die Benutzer erhalten aber einen Überblick über die Anzahl der verfügbaren Ergebnisseiten und können diese sequenziell über die Pfeil-Icons durchblättern oder über die direkte Eingabe der gewünschten Seitenzahl direkt auswählen. Abbildung \ref{fig:windowSearchResult} zeigt, wie in \emph{ANNIS 2.0} die Ergebnisliste in einem separaten Fenster dargestellt wird.

\begin{figure}[H]
	\centering
	\includegraphics*[width=1.0\textwidth]{figures/DA/windowSearchResult.jpg}
	\caption{Das Ergebnisfenster von \emph{ANNIS 2.0} stellt die wesentlichen Teile der einzelnen Fundstellen in Spalten angeordnet dar.}\label{fig:windowSearchResult}
\end{figure}

Zusätzlich zur Textansicht können für jede Fundstelle die verfügbaren Annotationsebenen angezeigt werden. Diese können durch Klicken auf das Expansionssymbol ($+$, $-$) angezeigt bzw. ausgeblendet werden. Die Benutzer werden aufgrund der schnellen Übersicht über die verfügbaren Informationen zusätzlich entlastet. Alle derzeit bekannten Annotationsebenen können in drei Klassen geordnet werden. \cite{futterleib2007} klassifiziert die Visualisierung von Annotationen in Partituren, Syntaxbäume und Koreferenzansichten die im nächsten Abschnitt beschrieben werden. Darüber hinaus ist möglich, \emph{ANNIS 2.0} um zusätzliche Visualisierungs-Module zu erweitern. Die dafür notwendigen Voraussetzungen und Schritte werden im Anhang \ref{sec:VisualisierungsModule} zusammengefasst. 

Da es nicht möglich ist, aus den \emph{PAULA-Inline}-Daten zu erkennen, ob eine bestimmte Annotationsebene als Partitur (Überdeckungsbeziehungen) oder Baum (Dominanzbeziehungen) dargestellt werden soll, wird diese Unterscheidung über die Konfiguration der Benutzerschnittstelle gesteuert.

\subsubsection{Darstellung von Spans in der Partituransicht}\label{sec:Darstellung.Partitur}

Spans werden üblicherweise als Partituren dargestellt. Dabei werden alle Annotationen mit ihrem Startpunkt der entsprechenden Länge als Spuren über einer  Basisebene (z.B. Tokenebene) angeordnet. Diese Darstellungsform kommt aus dem Bereich der Sprachkorpora, bei denen eine Zeitachse als Basis der Annotationen dient. Wie die nachstehende Abbildung \ref{fig:Partitur.Exmaralda} zeigt, bieten Partituransichten einen sehr guten Überblick über alle für einen bestimmten Ausschnitt verfügbaren Annotationen.

\begin{figure}[H]
	\centering
	\includegraphics*[width=1.0\textwidth]{figures/DA/PartiturExmaralda.jpg}
	\caption{Transkription in einem PRAAT-Textgrid \citep[vgl.][Abb. 4]{schmidt2005exmaralda})}\label{fig:Partitur.Exmaralda}
\end{figure}

\newpage
In \emph{ANNIS 2.0} wird die Darstellung der Parituransichten aus dem \emph{PAULA-Inline} (siehe Abschnitt \ref{sec:Korpussuche.Datenformat.PAULA-Inline}) in einem Servlet berechnet und als HTML ausgegeben. Diese Ausgabe wird ohne weitere Verarbeitungsschritte vom Web-Browser über ein IFrame\footnote{IFrames (abgel. von Inline Frames) sind HTML-Elemente zur Einbettung von anderen HTML-Dokumenten in das angezeigte Hauptdokument.} in die Ausgabe der Ergebnisliste integriert. Abbildung \ref{fig:PartiturProzess} illustriert diesen Prozess.

\begin{figure}[H]
	\centering
	\includegraphics*[width=0.70\textwidth]{figures/DA/PartiturProzess}
	\caption{Systematik des Prozesses zur Generierung von Partituransichten in \emph{ANNIS 2.0}}\label{fig:PartiturProzess}
\end{figure}

Die Partituransichten werden in \emph{ANNIS 2.0} als HTML-Tabellen dargestellt. Um diese dem Erscheinungsbild der Benutzerschnittstelle anzupassen, werden dafür die CSS-Klassen des \emph{ExtJS}-Frameworks verwendet. 

\begin{figure}[H]
	\centering
	\includegraphics*[width=0.9\textwidth]{figures/DA/ScreenshotPartitur.jpg}
	\caption{Partituransicht mehrerer Annotationsebenen in \emph{ANNIS 2.0}}\label{fig:PartiturScreenshot}
\end{figure}

Abbildung \ref{fig:PartiturScreenshot} zeigt das Endresultat der Entwicklungsarbeit anhand der Darstellung mehrerer Annotationsebenen eines gemeinsamen Namensraumes (Exmaralda). Dabei ist gut zu erkennen, dass die einzelnen Spans mit der zugrunde liegenden Token-Ebene aligniert sind und durch Berührung mit dem Mauszeiger (Hover) für jede Annotation die überdeckten Token sowie die vollständige Angabe des Schlüssel-Wert-Paares als Tooltip angezeigt werden können.

Hier wird insbesondere das Prinzip der Erwartungskonformität (siehe Abschnitt \ref{sec:DIN9241-10}) umgesetzt, da dieses Verhalten analog zu den einschlägigen Werkzeugen zur Erstellung solcher Partituren ist. Durch die Tooltips wird das Erlernen der Korpussuche unterstützt, da die Anzeigen der Anfrageschematik von \emph{AQL} entsprechen.

\subsubsection{Darstellung von Syntaxbäumen}\label{sec:Darstellung.Syntaxbäume}

Im Gegensatz zur Erstellung der Partituransichten ist die Visualisierung von Syntaxbäumen aufwendiger. Insbesondere erfordert die lesbare Anordnung der einzelnen Knoten und Kanten (Layout) mehr Anstrengung. Dieser Abschnitt beschäftigt sich mit dieser Teilaufgabe.

Analog zu den Partituransichten von flachen Annotationen werden die Syntaxbäume aus dem \emph{PAULA-Inline} berechnet. Die Ausgabe erfolgt im Bildformat \emph{JPEG}. Diese können dann über die URL des Servlets als Bildobjekte in die Ergebnislisten eingebunden werden. Die Berechnung des Layouts von Syntaxbäumen insbesondere durch mögliche kreuzende Kanten und der festen Reihenfolge der Tokenpositionen ist i.A. sehr aufwendig. Deshalb wird hierfür auf die Visualisierungsfunktionen der Open Source Software \emph{Graphviz} \citep{ellson2001graphviz} zurückgegriffen.

\begin{figure}[H]
	\centering
	\includegraphics*[width=0.70\textwidth]{figures/DA/GraphvizProzess.pdf}
	\caption{Systematik des Prozesses zur Generierung von Syntaxbäumen in \emph{ANNIS 2.0}}\label{fig:GraphvizProzess}
\end{figure}

Abbildung \ref{fig:GraphvizProzess} zeigt die für die Generierung von Syntaxbäumen notwendigen Arbeitsschritte. Das \emph{PAULA-Inline} wird beim Aufruf aus dem Cache des Web-Servers geladen, in das \emph{Graphviz}-Beschreibungsformat \emph{DOT} überführt und um Markierungen der Fundstellen angereichert. Mit Hilfe des auf dem Server installierten Kommandozeilenprogramms \emph{dot} wird das finale Bild errechnet. Die Ausgabe des Bildes erfolgt direkt über den Output-Stream des Servlets an den Web-Browser. Dort wird es über ein \emph{img}-Element in die HTML-Ausgabe der Ergebnisliste eingebettet.

Die Verwendung von \emph{DOT} macht es nicht möglich, die Syntaxbäume wie in Abbildung \ref{fig:tiger-example2} mit rechteckigen Kanten darzustellen. 

\begin{figure}[H]
	\centering
	\includegraphics*[width=0.9\textwidth]{figures/tiger-example1}
	\caption{Annotierter Beispielsatz aus dem \emph{TIGER}-Korpus}
	\label{fig:tiger-example2}
\end{figure}

Stattdessen werden die Kanten als direkte Verbindungen dargestellt. Da aber gerade im deutschsprachigen Raum die rechteckigen Kanten von \emph{TIGERSearch} bekannt sind, verletzt die Darstellung in \emph{ANNIS 2.0} in geringem Maße die von der DIN EN ISO 9241 (siehe Abschnitt \ref{sec:Normen}) verlangte Erwartungskonformität. Da diese Art von Syntaxbäumen jedoch analog zu \emph{TIGERSearch} ist, können alle notwendigen Informationen damit dargestellt werden. Das Ergebnis der Implementierung ist in Abbildung \ref{fig:SyntaxbaumScreenshot} zu sehen.

\begin{figure}[H]
	\centering
	\includegraphics*[width=0.9\textwidth]{figures/DA/ScreenshotSyntaxbaum.jpg}
	\caption{Syntaxbaum von \emph{TIGER}-Syntaxannotationen in \emph{ANNIS 2.0}}\label{fig:SyntaxbaumScreenshot}
\end{figure}

Gegenüber der HTML-Tabellen-Darstellung von \emph{ANNIS 1.0} stellt diese erwartungskonforme Darstellung eine deutliche Verbesserung der Lesbarkeit dar. Zusätzlich können die Bilddaten der Baumdarstellungen von den Benutzern separat gespeichert und ohne weitere Bearbeitungsschritte in bestehende Veröffentlichungsprozesse und den wissenschaftlichen Diskurs eingebunden werden. 

\newpage
\subsubsection{Darstellung von Koreferenz}\label{sec:Darstellung.Koreferenz}

Die dritte wichtige Art von Annotationen sind die Pointing Relations zur Beschreibung von Koreferenz. Im Gegensatz zu den Partituransichten und Baumdarstellungen ist es in diesem Fall nicht möglich, einen allgemeingültigen Mechanismus für die Darstellung unterschiedlicher Ebenen zu entwickeln. Grund dafür sind die stark unterschiedlichen Speichermöglichkeiten in \emph{PAULA-Inline}. Deswegen kann davon ausgegangen werden, dass für jede Art von Koreferenzannotation ein spezielles Visualisierungs-Modul implementiert werden muss.

Im Rahmen dieser Arbeit wurde ein Modul für die \emph{MMAX}-Annotationsebene umgesetzt. Allgemein ist es schwierig, Koreferenzen geeignet zu präsentieren, da dies i.A. nur für vollständige Texte sinnvoll ist. Eine Koreferenzansicht von einzelnen Textausschnitten ist in den meisten Fällen ungeeignet, da Beziehungen zu Textstellen außerhalb des dargestellten Textteils nicht gut darstellbar sind. Deshalb arbeitet das Visualisierung-Modul für \emph{MMAX} auf dem vollständigen Ausgangstext der jeweiligen Fundstellen.

Dieses Vorgehen erweist sich jedoch als kritisch, falls die Ausgangstexte sehr groß  (weit größer als 1000 Token) sind, da die Verarbeitungsdauer direkt von der Textlänge abhängt. Es kann aber davon ausgegangen werden, dass in sehr langen Texten Koreferenz nicht annotiert ist.

\begin{figure}[H]
	\centering
	\includegraphics*[width=1.0\textwidth]{figures/DA/Screenshot_Koreferenzansicht.jpg}
	\caption{Koreferenzansicht in \emph{ANNIS 2.0}}\label{fig:KoreferenzScreenshot}
\end{figure}

Abbildung \ref{fig:KoreferenzScreenshot} zeigt die \emph{MMAX}-Ansicht von \emph{ANNIS 2.0}. Diese ist in zwei Bereiche aufgeteilt. Auf der Linken Seite wird der vollständige Text angezeigt. Rechts daneben befindet sich für jeden Diskursreferenten ein Pull-Down-Menü welches alle Erwähnungen dieser Entität enthält sowie eine Checkbox zur Aktivierung der Hervorhebung im Text. Durch Umschaltung der Pull-Down-Menüs kann der Benutzer die beste Repräsentation der jeweiligen Diskursreferenten auswählen. Das ist notwendig weil aus den Daten nicht unbedingt erkennbar ist, welche Wortgruppe die Entität am besten beschreibt (''sie'' vs. ''Dagmar Ziegler'' vs. ''Finanzministerin''). Das Umschalten hat keine weitere Auswirkung auf die Darstellung. Durch Berühren einer Aktivierten Erwähnung im Text werden alle Erwähnungen des Diskursreferenten gesondert hervorgehoben. Damit bietet das System eine intuitive Präsentation von \emph{MMAX}-Koreferenz-Annotationen.

Mit den hier vorgestellten Visualisierungen für die unterschiedlichen Arten von Annotationsebenen bietet \emph{ANNIS 2.0} eine gute Basis für weitere Entwicklungen. Neben einer angemessenen Darstellung von Ergebnissen ist es jedoch auch wichtig, dass die Benutzer die untersuchten Phänomene effizient auffinden können. In diesem Fall stellt die Formulierung von geeigneten Anfragen wohl die größte Hürde dar. Um die Benutzer hierbei zu unterstützen, stellt \emph{ANNIS 2.0} einen grafischen Anfragen-Editor zur Verfügung.

\subsection{Der grafische Anfragen-Editor \emph{ANNIS.build}}\label{sec:ANNIS.build}

Auch wenn die Anfragesprache \emph{AQL} die intuitive Beschreibung des Gesuchten ermöglicht (siehe Abschnitt \ref{sec:AQL.Usability}), ziehen es viele Benutzer vor, die Anfragen in grafischer Form einzugeben. Gerade komplexe Anfragen über den Bereich der Syntax werden mit dieser Eingabemethode deutlich vereinfacht.

Der grafische Anfragen-Editor \emph{ANNIS.build} bildet die Grundelemente von \emph{AQL} als Objekte ab, die dann bequem mit der Maus und der Tastatur bearbeitet und verknüpft werden können.

\begin{figure}[H]
	\centering
	\includegraphics*[width=1.0\textwidth]{figures/DA/ANNIS_build.jpg}
	\caption{Der grafische Anfragen-Editor \emph{ANNIS.build} ermöglicht es, die Grund-elemente von \emph{AQL} visuell zu bearbeiten und zu verknüpfen.}\label{fig:ANNIS_build}
\end{figure}

Die Abbildung \ref{fig:ANNIS_build} zeigt, wie die Knoten- und Kantendefinitionen einer komplexen Anfrage als verknüpfte Fenster dargestellt werden. Die resultierende \emph{AQL}-Anfrage wird sofort in das Feld für die Suchanfrage übertragen und die Ergebnismenge berechnet. Die Übertragung einer \emph{AQL}-Anfrage in den Editor ist aufgrund der benötigten Layout-Routinen nicht implementiert.

\emph{ANNIS.build} ist damit der modernste grafische Anfrageneditor innerhalb einer web-basierten Korpussuche. Er greift bekannte Konzepte von \emph{TIGERSearch} auf und macht sie für verschiedenartige Korpora nutzbar. Eine lokale Installation beim Benutzer ist nicht notwendig. Diese Umsetzung ist allerdings nur unter Verwendung von komplexem, client-seitig ausgeführtem Code möglich. Eine fliessende Einbettung in die Benutzerschnittstelle ist weiterhin nur unter Zuhilfenahme von \emph{JavaScript} möglich.

\subsection{Zugangskontrolle}\label{sec:Authentifizierung}

Eine wichtige Voraussetzung für die Akzeptanz von \emph{ANNIS 2.0} ist die Kontrolle des Zugriffs auf die verfügbaren Korpora. Dabei geht es zum einen um die Wahrung von Lizenzvereinbarungen bei der Verwendung urheberrechtlich geschützter Korpusdaten und zum anderen um die Zugangskontrolle für Korpusdaten, die sich noch in der Entwicklung befinden oder aus anderen Gründen nur einem beschränkten Nutzerkreis zugänglich sein sollen.

Die zwei dafür notwendigen Komponenten sind Authentifizierung und Autorisierung. Authentifizierung bezeichnet den Prozess der Identifikation des Benutzers mit \emph{Benutzername} und \emph{Passwort}. Sind die bei der Anmeldung vom Benutzer korrekt eingegeben worden, wird eine neue \emph{Sitzung} angelegt. Der Zugriff auf die einzelnen Korpora wird über den \emph{Autorisierungs}mechanismus kontrolliert.

\emph{ANNIS 2.0} kann an ein \emph{LDAP}\footnote{\emph{LDAP}: Lightweight Directory Access Protocol}-Server angeschlossen werden. Dieser \emph{LDAP}-Dienst kann sowohl lokal auf dem Web-Server als auch zentral installiert sein und verwaltet alle notwendigen Benutzer und Benutzergruppen. Die einzige Voraussetzung ist, dass dieser von der Web-Applikation aus erreichbar ist. Gruppen können andere Gruppen und Benutzer enthalten. Um die Zugriffe auf die einzelnen Korpora zu verwalten, wird für jedes Korpus im \emph{LDAP} eine korrespondierende Gruppe angelegt. Nur die darin enthaltenen Gruppen und deren Benutzer haben die notwendige Berechtigung zur Suche in diesem Korpus.

Die Authentifizierung wird direkt durch die Anmeldung der Benutzer am \emph{LDAP} Server vorgenommen. Im Gegensatz dazu findet die Autorisierung in den Servlets von \emph{ANNIS 2.0} statt. Dafür werden die bei der Anmeldung geladenen und in der \emph{Sitzung} gespeicherten Sicherheitsinformationen für den angemeldeten Benutzer bei jedem Zugriff auf Korpusinformationen oder -daten ausgewertet.

\subsection{Zitieren von Suchergebnissen}\label{sec:Zitieren}

Wissenschaftliche linguistische Veröffentlichungen beinhalten oft Verweise auf die Ergebnisse von durchgeführten Korpussuchen. Dabei wird oft nur auf die Quelle und ggf. auf einige ausgewählte Fundstellen verwiesen. Oft fehlt jedoch die Angabe der Suchanfrage, um diese auf Korrektheit in Bezug auf die gewünschte Suchbedeutung überprüfen zu können. So ist es z.B. durchaus vorstellbar, dass eine Suchanfrage das Gesuchte nur unvollständig oder vielleicht gar nicht erfasst. Insbesondere Schlussfolgerungen, die aus statistischen Berechnung über den Ergebnissen von Suchanfragen getroffen werden, sind für diese Kritik anfällig. Darüber hinaus zeigen die Ergebnisse der Nutzerbefragung (Frage 3, Tabelle \ref{tab:Frage3}), dass ein modernes Korpussystem sich einfach in den wissenschaftlichen Diskurs einbeziehen lassen muss. 

Dafür erlaubt \emph{ANNIS 2.0} dem Benutzer, eine eindeutige Kennung für jede durchgeführte Suche berechnen zu lassen. Diese Kennung enthält neben der Suchanfrage auch Metainformationen zur Reproduktion der gewünschten Suchergebnisse. Diese sollten möglichst in menschenlesbarer Form serialisiert\footnote{Serialisieren ist das Kodieren von Daten in eine Sequenz von Zeichen.} werden. Im Gegenzug muss es möglich sein, aus den serialisierten Daten alle für die Suchanfrage notwendigen Informationen zu rekonstruieren und die Sucheingabe automatisiert vorzunehmen.

Eine Suchanfrage in \emph{ANNIS 2.0} besteht aus folgenden notwendigen Komponenten:

\begin{itemize}
	\item{\emph{AQL} Suchanfrage (AQL)}
	\item{Selektierte Korpora (CIDS)}
	\item{Suchkontext links (CLEFT)}
	\item{Suchkontext rechts (CRIGHT)}
\end{itemize}

Zur Serialisierung werden die Komponenten als Key-Value-Paare mit Kommata separiert konkateniert. Damit die Suchanfragen als URL in Veröffentlichungen eingebettet werden können, werden diese gemäß RFC2396\footnote{Uniform Resource Identifiers (URI): Generic Syntax (RFC 2396); http://www.rfc-editor.org/rfc/rfc2396.txt} kodiert. Die selektierten Korpora werden durch deren eindeutige ID gekennzeichnet.

So beschreibt beispielsweise der folgende Link die Suche auf dem Server\\  korpling.german.hu-berlin.de in den Korpora TIGER\_070115--inline (ID 2) und\\ pcc176--inline (ID 6) nach ''Häuser'' mit Anzeigekontext 5 für Links und Rechts:

\begin{center}
https://korpling.german.hu-berlin.de/Annis/Cite/\\AQL(\%22H\%8Auser\%22),CIDS(2,6),CLEFT(5),CRIGHT(5)
\end{center}

Die Entgegennahme solcher Zitierungen wird über das auf \emph{/Annis/Cite} hinterlegte \emph{Java-Servlet} \emph{annis.frontend.servlets.CitationServlet} und \emph{JavaScript} Code in der Datei
\emph{windowSearch.js} realisiert. Das Servlet nimmt dabei nur die Zitierungsinformationen entgegen und speichert diese unverarbeitet in einem \emph{Cookie}\footnote{\emph{Cookies} sind eine Möglichkeit, Informationen auf den Computern von Webbenutzern zu speichern.}. Nach erfolgreicher Authentifizierung des Benutzers werden diese Daten via \emph{JavaScript} ausgelesen und das Eingabefenster für die Suchanfrage entsprechend vorausgefüllt. Danach wird die Anfrage zur Ausführung gebracht und das Ergebnisfenster geöffnet.

Damit die Umsetzung der Zitierung auch nachhaltig von den Benutzern verwendet werden kann, müssen alle zukünftigen Updates der \emph{ANNIS} Suchschnittstelle die Abwärtskompatibilität zu diesem Zitiersystem gewährleisten. Ebenso dürfen sich die Identifikationsnummern einmal veröffentlichter Korpora nicht ändern, was auch bei der Versionierung der Korpora berücksichtigt werden muss.

\newpage
\section{Diskussion}\label{sec:Diskussion}

Mit der hier vorgestellten Suchschnittstelle bietet sich die Chance, ein zukunftsweisendes und benutzerorientiertes Korpussystem zu implementieren. Insbesondere mit der Verwendung moderner Webtechnologien wie modularer \emph{Java-Servlets} (siehe Abschnitt \ref{sec:Web-Applikation}) oder \emph{AJAX} (siehe Abschnitt \ref{sec:AJAX}) orientieren sich Erweiterungen und Anpassungen an aktuellen Bedürfnissen.

\emph{ANNIS 2.0} befindet sich derzeit in einem Entwicklungszustand und besitzt noch nicht alle gewünschten Grundfunktionen. Deshalb werden in diesem Kapitel sinnvolle Erweiterungen vorgestellt. Diese können und sollen als Anregung für die Weiterentwicklung der Suchschnittstelle dienen.

\subsection{Das aufgabensensitive Hilfesystem \emph{ANNIS.help}}

Zusätzlich zu der Usability Heuristik ''Hilfe und Dokumentation'' (siehe Abschnitt \ref{sec:Heuristiken}) fordern auch die Ergebnisse der im Rahmen dieser Arbeit gemachten Untersuchung --besonders die Fragen 18 und 19 (siehe Tabellen \ref{tab:Frage18} und \ref{tab:Frage19})--, dass neben Workshops und schriftlicher Dokumentation ein geeignetes Hilfesystem das dritte wichtige Hilfsmittel für das Erlernen von modernen Korpussystemen ist. 

\emph{ANNIS.help} sollte ein wegweisendes, aufgabensensitives Hilfesystem sein. Es ist vorstellbar, dass sowohl Hilfetexte als auch zielorientierte Tutorials, abhängig von erkannten Sucheingaben, angezeigt werden. Ebenso kann aus einer Tutorial-Funktion die Eingabemaske mit Hilfe von \emph{JavaScript} bedient und eingestellt werden. Der Benutzer kann dann vorgeschlagene Eingaben durch Hyperlinks aus dem Hilfstext heraus übernehmen und so das gesetzte Ziel effizient erreichen. 

Die Korpussuchen die auf der Plattform von \citep{davies2005ijcl} basieren, bieten eine Übersicht von Beispielen an, die alle notwendigen Eingaben über \emph{JavaScript} automatisch vornehmen kann. Abbildung \ref{fig:View.Help} zeigt, wie die Hilfe nach dem Klick auf das erste Syntaxbeispiel (Suchwort ''mysterious'') die Suchmaske gefüllt und die Ergebnisse angezeigt hat.

\begin{figure}[H]
	\centering
	\includegraphics*[width=1.0\textwidth]{figures/DA/Screenshot-View-Hilfe.jpg}
	\caption{Das \emph{BYU-BNC}-Korpus stellt ein interaktives Hilfesystem zur Verfügung.}
	\label{fig:View.Help}
\end{figure}

Ein derartiges Hilfesystem kann aber nur das Ergebnis hoher redaktioneller und didaktischer Anstrengungen sein. Die Grenzen der technischen Machbarkeit sind hier vergleichsweise gering. So bietet die schon implementierte Zitierfunktion eine Grundlage für die Verlinkung von Beispielanfragen in Online-Dokumentation.

\subsection{Datenexport mit \emph{ANNIS.export}}

In vielen Gesprächen mit Korpusnutzern hat sich gezeigt, dass insbesondere für statistische Auswertungen mit externen Programmen wie z.B. R\footnote{R Homepage: http://www.r-project.org/}, SPSS\footnote{SPSS Homepage: http://www.spss.com/} oder Microsoft Excel eine Exportfunktion benötigt wird.

Für \emph{ANNIS 2.0} bietet es sich an, die Exportfunktionalität über einen Warenkorb, wie er aus Online-Shops bekannt ist, zu lösen. Vollständige Suchergebnisse oder einzelne Fundstellen können so gesammelt und anschließend mit den gewünschten Optionen auf den Computer der Benutzer heruntergeladen werden. Als Exportformate bieten sich CSV\footnote{CSV: Comma Separated Values. Ein Austauschformat für tabellarische Daten.}  für Tabellenansichten oder ein XML-basiertes Austauschformat wie \emph{PAULA [ggf. Inline]} (siehe Abschnitt \ref{sec:Korpussuche.Datenformat.PAULA-Inline}) für Baumstrukturen an.

Diese Exportfunktionen werden vollständig in der Web-Applikation implementiert und sollten hinsichtlich der Bedürfnisse menschlicher Benutzer optimiert werden. Für einen maschinellen Zugriff ist der Export über die Benutzerschnittstelle deshalb ungeeignet. Alle automatischen Interaktionen mit \emph{ANNIS} sollten daher direkt mit dem \emph{ANNIS-Service} etabliert werden. Nähere Informationen dazu sind in Abschnitt \ref{sec:DDDQuery} zu finden.

\subsection{Datenimport mit \emph{ANNIS.import}}

Alle erhobenen Daten müssen, bevor diese geeignet durchsucht werden können, in das Suchbackend von \emph{ANNIS} übertragen werden. Dabei werden alle Informationen kopiert und mit berechneten Metadaten für eine optimierte Suche angereichert. Nicht selten verlangt dieser Prozess die Konvertierung der Daten in das eigentliche Importformat \emph{PAULA-Stand-Off} (vgl. Abschnitt \ref{sec:Korpussuche.Datenformat.PAULA-Inline}). Liegen die Daten im Importformat vor, können diese an die Importfunktion des Suchbackends übergeben werden.

Damit der Importprozess vom Nutzer durchgeführt werden kann, muss dieser in die Benutzerschnittstelle \emph{ANNIS 2.0} integriert werden. Hierbei kann es sich um eine einfache Upload-Funktionalität handeln. Vorzugsweise sollte der Benutzer in der Lage sein, vollständige Korpora in das System einzuspielen. Hierfür wären z.B. \emph{ZIP}-komprimierte \emph{PAULA}-Dateien sehr gut geeignet.

Sobald \emph{ANNIS} eine derartige Importfunktionalität besitzt, können Benutzer selbsterstellte Korpora unkompliziert in das System übertragen, diese durchsuchen und anderen Benutzern zur Verfügung stellen. Die Erstellung der Annotationen bleibt jedoch ein separater Prozess, für den unterschiedliche Werkzeuge notwendig sein können. Um auch diesen Arbeitsschritt einfacher zu gestalten, ist die Integration eines web-basierten Annotationswerkzeuges in \emph{ANNIS} denkbar. Die Vorteile und Voraussetzungen dafür werden im folgenden Abschnitt vorgestellt.

\subsection{Online Annotation mit \emph{ANNIS.augment}}

\emph{ANNIS 2.0} ist als Multi-Korpus-Plattform ideal dazu geeignet, zusätzlich zur Suche in Korpora auch den Import von Rohdaten und deren Annotationen zu unterstützen. Es ist vorstellbar, dass Benutzer direkt im Web-Interface bestehende Annotationen bearbeiten oder neue Annotationen und Annotationsebenen hinzufügen können. Die Voraussetzungen für diese Interaktionsform sind jedoch erheblich.

Zunächst muss dafür das Datenbackend um Edierfunktionen erweitert werden. Die punktuelle Änderung eines Annotationswertes ist dabei als unproblematisch einzuschätzen. Jedoch das Verschieben oder Hinzufügen eines Knotens in einer Baumstruktur, wie sie u.a. in Syntaxannotationen zu finden ist, erfordert konzeptionellen und handwerklichen Aufwand.

Bei der Erweiterung der Benutzerschnittstelle wird der Hauptaufwand bei der Entwicklung und Evaluation geeigneter Editoren liegen. Besonders das rein web-basierte Edieren von Graphen stellt eine große Herausforderung dar.

Dem übergeordnet müssen vorerst konzeptionelle Fragen zur Versionierung und zum Release-Management der Daten geklärt werden. Da alle öffentlichen und halböffentlichen Daten potentiell für wissenschaftliche Veröffentlichungen als Belegquellen dienen, sind angepasste Versionierungsmechanismen von großer Wichtigkeit.

Es bleibt anzumerken, dass es bereits web-basierte Werkzeuge zur Annotation gibt. Beispiele dafür sind z.B. die \emph{Perseus Digital Library}\footnote{\emph{Perseus Digital Library}: http://www.perseus.tufts.edu/} und das \emph{PROIEL}\footnote{\emph{PROIEL} Korpus: http://www.hf.uio.no/ifikk/proiel/corpus.html}. Diese Lösungen befinden sich derzeit noch in einer sehr rudimentären Entwicklungsstufe und bieten keine grafisch orientierten Manipulationsmöglichkeiten an.

\subsection{Offline Suchen mit \emph{ANNIS.local}}\label{sec:ANNIS.local}

In einigen Forschungskontexten ist korpuslinguistische Arbeit ohne Internetzugang denkbar. Um auch dort die Möglichkeiten der Korpussuche mit \emph{ANNIS} nutzen zu können, sollte es möglich sein, das System auch auf einem mobilen Computer ohne Anbindung an einen Server betreiben zu können. Die erste Version von \emph{ANNIS} unterstützt die lokale Offline-Suche mit der Softwareversion \emph{ANNIS.local}.

Für die lauffähige Version von \emph{ANNIS 2.0} zu erhalten, müssen derzeit noch eine Vielzahl notwendiger Komponenten separat installiert und konfiguriert werden. Um diesen Schritt so einfach wie möglich zu gestalten, sollte dafür ein automatisches Installationspaket erstellt werden. Es ist zu erwarten, dass durch zahlreiche lokale Installationen die Anfragen an die Supportgruppe ansteigen wird.

\subsection{Der \emph{ANNIS}-Service als Teil eines linguistischen Netzwerks}

Die \emph{ANNIS 2.0} Web-Applikation kann mit jedem \emph{Java-RMI}-Service interagieren, der die \emph{ANNIS-Service}-Dienstschnittstellen und Datenobjekte implementiert. Dadurch können angepasste Implementierungen der Suche sowohl mit \emph{ANNIS QL} als auch mit anderen Anfragesprachen angebunden werden.

Die zentrale Rolle bei der Ausführung der Suchanfragen spielt der \emph{Java RMI} Service, basierend auf \emph{DDDQuery} (siehe Abschnitt \ref{sec:DDDQuery}). Einziger Klient dieses Dienstes ist vorerst die Web-Anwendung \emph{ANNIS 2.0}. Hier können in Zukunft auch andere Anwendungen als Klienten auftreten.

Dafür muss der Service für die betreffenden Netzwerkhosts freigegeben werden. Zusätzlich wird es notwendig, alle bis dahin implementierten Authentifizierungs- und Autorisierungsmechanismen auch an dieser Stelle einzubinden.

Ein solcher Dienst wäre ein entscheidender Schritt für ein weites Netz linguistischer Dienste. Ebenso können andere Services wie z.B. Ontologien \citep[vgl.][]{chiarcosXXXXonthology} genutzt oder sogar Metasuchen über mehreren \emph{ANNIS} Diensten implementiert werden. 

\section{Fazit}

Mit der in dieser Arbeit vorgestellten Benutzerschnittstelle ANNIS 2.0 wurde die Grundlage für ein neuartiges Korpussystem geschaffen, das vor allem die Vorteile birgt, unterschiedliche (Sub-) Korpora zusammenzuführen und mit einer Suchanfrage zu durchsuchen, die Fundstellen ganzheitlich vergleichend analysieren zu können und die Arbeitsergebnisse auf einfache Weise in bestehende Arbeitsprozesse einzubinden.

%Bei der Erstellung dieser Web-Anwendung wurde vor allem auf einen modularen Aufbau Wert gelegt, um ..., und es wurde darauf geachtet, dass die Darstellungsmöglichkeiten (gemäß unterschiedlichen Nutzeranforderungen) möglichst vielseitig sind. Als ein weiterer positiver Effekt dessen können Erweiterungen oder Änderungen leicht vorgenommen werden.


Bei der Erstellung dieser Web-Anwendung wurde vor allem auf einen modularen Aufbau Wert gelegt, um weitere Visualisierungs-Methoden wie z.B. Audio-, Video- oder Bildinformationen der Quelldaten an das System (siehe Anhang \ref{sec:VisualisierungsModule}) anfügen zu können und es wurde darauf geachtet, dass auch die bestehenden Darstellungsmöglichkeiten (gemäß unterschiedlichen Nutzeranforderungen) möglichst vielseitig sind. Als ein weiterer positiver Effekt dessen können Erweiterungen oder Änderungen leicht vorgenommen werden, wovon auch Projekte ausserhalb des Sonderforschungsbereiches (SFB 632) profitieren können.

%Durch die vielseitigen Darstellungsmöglichkeiten und den modularen Aufbau dieser Web-Anwendung können Erweiterungen oder Änderungen leicht vorgenommen werden. Besonders durch die Möglichkeit neue Visualisierungs-Module an das System (siehe Anhang \ref{sec:VisualisierungsModule}) anzufügen, können auch andere Projekte, die mit \emph{ANNIS 2.0} arbeiten, neue Arten von Annotationen einbetten. Denkbar sind hier z.B. Audio-, Video- oder Bildinformationen der Quelldaten.

Die verwendeten Technologien sind sehr aktuell und verlangen auch von erfahrenen Entwicklern das Erlernen neuer Konzepte und Herangehensweisen. Dadurch ist es aber möglich, web-basierte Benutzerschnittstellen zu entwickeln, die zentralisierte Lösungen mit den Vorteilen von Desktop-Anwendungen zu verknüpfen.

Auch wenn die durchgeführte Nutzerbefragung durch die geringe Teilnehmerzahl keine allgemeinen quantitativen Aussagen zulässt, bestätigt sie jedoch, dass die heutigen Korpusnutzer derzeit auf eine große Bandbreite von unterschiedlichen Korpusdaten in einem einheitlichen System zugreifen wollen. In diesem Punkt liegen die Stärken von \emph{ANNIS 2.0}. Mit der Verwendung der flexiblen Suchplattform \emph{DDDQuery} (Abschnitt \ref{sec:DDDQuery}) und der benutzerfreundlichen Anfragesprache \emph{AQL} (Abschnitt \ref{sec:AQL}) besitzt die Korpuslinguistik nun ein Werkzeug, mit dem zukünftige Anforderungen effizient und lösungsorientiert umgesetzt werden können.

\newpage
\addcontentsline{toc}{section}{Literatur}
\bibliographystyle{plainnat_german}
\bibliography{literature}

\textbf{Alle im Text aufgeführten URLs wurden am 29. August 2008 das letzte mal auf deren Erreichbarkeit und den verfügbaren Inhalt in Bezug auf diese Arbeit überprüft.}

\newpage
\addcontentsline{toc}{section}{Selbständigkeitserklärung}
\section*{Selbständigkeitserklärung}

Ich versichere, dass ich die vorliegende Arbeit selbständig verfasst und nur die angegebenen Quellen und Hilfsmittel verwendet habe. Wörtlich oder dem Sinn nach aus anderen Werken entnommene Stellen sind unter Angabe der Quellen kenntlich gemacht. \\ \\ \\ \\ \\ \\

\begin{tabular}{l r}
	Ort, Datum & Unterschrift
\end{tabular}

\addcontentsline{toc}{section}{Einverständniserklärung}
\section*{Einverständniserklärung}

Ich erkläre hiermit mein Einverständnis, dass die vorliegende Arbeit in der Bibliothek des Institutes für Informatik der Humboldt-Universität zu Berlin ausgestellt werden darf.\\ \\ \\ \\ \\ \\

\begin{tabular}{l r}
	Ort, Datum & Unterschrift
\end{tabular}


\newpage
\section{Anhang}
\subsection{Erstellung zusätzlicher Visualisierungs-Module}\label{sec:VisualisierungsModule}

\emph{ANNIS 2.0} kann um zusätzliche Visualisierungs-Module erweitert werden. Diese werden als \emph{Java}-Klassen implementiert und müssen die abstrakte Basisklasse\newline \emph{annis.frontend.servlets.visualizers.Visualizer} erweitern. 

Prinzipiell muss dafür die Methode \emph{writeOutput(Writer)} implementiert werden. Diese kann dann die Klasseneigenschaft \emph{super.paula} auslesen, parsen und die erwünschte Ausgabe auf die übergebene \emph{Writer}-Variable schreiben. Die Detailinformationen zu allen Methoden dieser Basisklasse und deren Bedeutungen sind Teil der API-Dokumentation auf der beiliegenden CD (/AnnisWEB/doc).

Die Instantiierung aller Visualisierungs-Module geschieht über das  Servlet \\\emph{annis.frontend.servlets.VisualizerServlet}. Das folgende \emph{Java}-Code-Listing \ref{lst:VisualizerUsage} soll die Verwendung illustrieren.\\

\lstset{language=Java,frame=ltrb,framesep=5pt,
 basicstyle=\footnotesize,
 keywordstyle=\ttfamily\color{Keyword},
 identifierstyle=\ttfamily\color{Identifier}\bfseries, 
 commentstyle=\color{Comment},
 stringstyle=\ttfamily,
 showstringspaces=true,
 caption=Prinzip der Anwendung der Visualizer-Implementierungen,
 label=lst:VisualizerUsage}

\begin{lstlisting}
//Instantiate and feed visualizer
Visualizer visualizer = (Visualizer) 
	classLoader.loadClass(visualizerClassName).newInstance();
visualizer.setNamespace(namespace);
visualizer.setMarkableMap(markableMap);
visualizer.setPaula(paula);

//Set parameters on ServletResponse			
response.setCharacterEncoding(
		visualizer.getCharacterEncoding()
	);
response.setContentType(visualizer.getContentType()); 
    
visualizer.writeOutput(response.getWriter());
\end{lstlisting}

Die beispielhafte Implementierung eines Visualizers befindet sich auf der beiliegenden CD-ROM (/AnnisWEB/src/annis/frontend/servlets/visualizers/ExampleVisualizer.java). Sie enthält den Quelltext aus dem folgenden Listing \ref{lst:ExampleVisualizer}.\\

\lstset{language=Java,frame=ltrb,framesep=5pt,
 basicstyle=\footnotesize,
 keywordstyle=\ttfamily\color{Keyword},
 identifierstyle=\ttfamily\color{Identifier}\bfseries, 
 commentstyle=\color{Comment},
 stringstyle=\ttfamily,
 showstringspaces=false,
 caption=annis/frontend/servlets/visualizers/ExampleVisualizer.java,
 label=lst:ExampleVisualizer}

\lstinputlisting{figures/DA/ExampleVisualizer.java}

\newpage
\subsection{Inhalt des Fragebogens}\label{sec:Fragebogen.Inhalt}

In diesem Abschnitt ist der vollständige Inhalt der Benutzerbefragung in Textform aufgeführt. Einfachauswahlen sind mit dem Symbol $\bigcirc$ und Mehrfachauswahlen durch $\Box$ gekennzeichnet. Freitext-Eingaben sind mit \underline{\ \ \ \ \ \ \ \ \ \ \ \ \ \ \ } markiert. Aus technischen Gründen wird in die Rating-Skalen der Online-Version die Antwortoption ''keine Antwort'' eingefügt. Diese Funktionsweise ist programmintern und konnte im Rahmen dieser Arbeit nicht verändert werden.
	
Die Fragen 10 und 11 wurden  nur dann angezeigt, wenn Frage 9 mit ''ja'' beantwortet wurde. Die Ergebnisse werden in Abschnitt \ref{sec:Nutzerbefragung.Auswertung} vorgestellt und diskutiert.

Der Begrüßungstext wird auf einer separaten Einstiegsseite angezeigt und hat folgenden Wortlaut:
\begin{quote}
Herzlich willkommen zur Nutzerbefragung zur Korpussuche. Es werden Ihnen Fragen zu Ihren Erfahrungen mit der Korpusauswertung gestellt. Die meisten davon sind Selbsteinschätzungsfragen. Hier bitten wir Sie, so objektiv es geht zu antworten. Wenn Sie eine Frage nicht beantworten können oder wollen, lassen Sie diese einfach frei. 

Die Ergebnisse dieser Befragung werden direkten Einfluss auf das vom SFB632 Projekt D1 (http://www.sfb632.uni-potsdam.de/projects\_d1ger.html) entwickelte Korpussystem ANNIS 2.0 nehmen. Sie geben uns mit Ihren Antworten die Möglichkeit ein Werkzeug zu erschaffen, das Ihre Bedürfnisse erfüllt und Sie in Ihrer Arbeit optimal unterstützt. 

Wir würden uns freuen, wenn Sie noch anderen Interessierten den Link auf diese Umfrage zukommen lassen würden. Mehr Teilnehmer sorgen für ein besseres Bild über unsere Nutzer. 

Vielen Dank für Ihre Teilnahme.
\end{quote}

Die Fragen werden wie im folgenden angegeben auf einer zweiten Seite vollständig präsentiert. Die Befragten konnten den Fragebogen jederzeit abschließen oder durch schließen des Browser-Fensters abbrechen.

	\begin{enumerate}
	\item{
		\textbf{Wie lange wissen Sie schon von Korpora?}\\*
		
			\begin{tabular}{c l}
				$\bigcirc$ & < 3 Monate \\
				$\bigcirc$ & 3-6 Monate \\
				$\bigcirc$ & 6-9 Monate \\
				$\bigcirc$ & 9-12 Monate \\
				$\bigcirc$ & 1-2 Jahre \\
				$\bigcirc$ & 2-3 Jahre \\
				$\bigcirc$ & > 3 Jahre
			\end{tabular}
	}
	
	\item{
		\textbf{Wie lange benutzen Sie schon Korpora für Ihre Arbeit(en)?}\\*

			\begin{tabular}{c l}
				$\bigcirc$ & < 3 Monate \\
				$\bigcirc$ & 3-6 Monate \\
				$\bigcirc$ & 6-9 Monate \\
				$\bigcirc$ & 9-12 Monate \\
				$\bigcirc$ & 1-2 Jahre \\
				$\bigcirc$ & 2-3 Jahre \\ 
				$\bigcirc$ & > 3 Jahre
			\end{tabular}
	}
	
	\item{
		\textbf{Wie wichtig ist die Auswertung von Korpusdaten für Ihre Veröffentlichungen?}\\*
		\begin{center}
			\begin{tabular}{c | c | c | c}
				nicht wichtig & weniger wichtig & wichtig & sehr wichtig \\
				$\bigcirc$ & $\bigcirc$ & $\bigcirc$ & $\bigcirc$
			\end{tabular}
		\end{center}
	}
	
	\item{
		\textbf{Benötigen Sie die Möglichkeit, Korpusdaten offline (d.h. lokal auf Ihrem Computer) umfangreich auswerten zu können?}\\*
		\begin{center}
			\begin{tabular}{c | c | c}
					ja & nein & unsicher \\
					$\bigcirc$ & $\bigcirc$ & $\bigcirc$
			\end{tabular}
		\end{center}
	}
	
	\item{
		\textbf{Wie stark interessieren Sie sich für folgende Korpusarten?}\\*
		\begin{center}
			\begin{tabular}{l c c c  c c c c}
					 & (gar nicht) & 1 & 2 & 3 & 4 & 5 & (sehr) \\
					(Synchrone) Textkorpora & &  $\bigcirc$ & $\bigcirc$ & $\bigcirc$ & $\bigcirc$ & $\bigcirc$ \\ \hline
					Korpora gesprochener Sprache & &  $\bigcirc$ & $\bigcirc$ & $\bigcirc$ & $\bigcirc$ & $\bigcirc$ \\ \hline
					Multimodale Korpora & &  $\bigcirc$ & $\bigcirc$ & $\bigcirc$ & $\bigcirc$ & $\bigcirc$ \\ \hline
					Parallelkorpora & &  $\bigcirc$ & $\bigcirc$ & $\bigcirc$ & $\bigcirc$ & $\bigcirc$ \\ \hline
					Historische Korpora & &  $\bigcirc$ & $\bigcirc$ & $\bigcirc$ & $\bigcirc$ & $\bigcirc$
			\end{tabular}
		\end{center}
	}
	
	\item{
		\textbf{Wie oft nutzen Sie die folgenden Annotationsebenen in Ihren Auswertungen?}\\*
		\begin{center}
			\begin{tabular}{l c c c  c c c c}
					 & (gar nicht) & 1 & 2 & 3 & 4 & 5 & (sehr) \\
					Wortarten/Lemmatisierung & &  $\bigcirc$ & $\bigcirc$ & $\bigcirc$ & $\bigcirc$ & $\bigcirc$ \\ \hline
					Morphologie & &  $\bigcirc$ & $\bigcirc$ & $\bigcirc$ & $\bigcirc$ & $\bigcirc$ \\ \hline
					Syntaktische Annotation (Baumbanken) & &  $\bigcirc$ & $\bigcirc$ & $\bigcirc$ & $\bigcirc$ & $\bigcirc$ \\ \hline
					Semantische Annotation & &  $\bigcirc$ & $\bigcirc$ & $\bigcirc$ & $\bigcirc$ & $\bigcirc$ \\ \hline
					Informationsstruktur & &  $\bigcirc$ & $\bigcirc$ & $\bigcirc$ & $\bigcirc$ & $\bigcirc$ \\ \hline
					Koreferenz & &  $\bigcirc$ & $\bigcirc$ & $\bigcirc$ & $\bigcirc$ & $\bigcirc$
			\end{tabular}
		\end{center}
	}
	
	\item{
		\textbf{Welche(s) Betriebssystem(e) verwenden Sie für Ihre Arbeit?}\\*

			\begin{tabular}{c l}
				$\Box$ & Windows 98/ME \\
				$\Box$ & Windows XP \\
				$\Box$ & Windows Vista \\
				$\Box$ & Linux \\
				$\Box$ & MacOS 9 \\
				$\Box$ & MacOS X \\
				$\Box$ & Sun Solaris \\
				$\Box$ & \underline{\ \ \ \ \ \ \ \ \ \ \ \ \ \ \ }
			\end{tabular}
	}
	
	\item{
		\textbf{Nennen Sie drei Korpussysteme, mit denen Sie arbeiten. Beginnen Sie mit dem wichtigsten.}\\*
		
			\begin{tabular}{l l}
				1. & \underline{\ \ \ \ \ \ \ \ \ \ \ \ \ \ \ } \\
				2. & \underline{\ \ \ \ \ \ \ \ \ \ \ \ \ \ \ } \\
				3. & \underline{\ \ \ \ \ \ \ \ \ \ \ \ \ \ \ }
			\end{tabular}
	}
	
	\item{
		\textbf{Schätzen Sie Ihre Fertigkeiten im Umgang mit den soeben genannten Systemen ein.}\\*
		\begin{tabular}{l p{300pt}}
			\emph{Experte} & Ich habe das System, die Datenbasis und die Anfragesprache verstanden. Ich wüsste sogar sinnvolle Änderungen/Erweiterungen. \\
			\emph{Umfangreich} & Ich benötige keine Dokumentation mehr, um ein Vielzahl verschiedenartige Anfragen zu machen. \\
			\emph{Ausreichend} & Ich kann einfache Anfragen ohne Dokumentation schreiben. Für kompliziertere Anfragen benötige ich jedoch die Dokumentation. \\
			\emph{Rudimentär} & Ich weiss wie Anfragen prinzipiell funktionieren, benötige jedoch in nahezu jedem Fall Hilfe.
		\end{tabular}
		
		\begin{center}
			\begin{tabular}{r c c c c}
				& Rudimentär & Ausreichend & Umfangreich & Experte \\
				1. & $\bigcirc$ & $\bigcirc$ & $\bigcirc$ & $\bigcirc$ \\ \hline
				2. & $\bigcirc$ & $\bigcirc$ & $\bigcirc$ & $\bigcirc$ \\ \hline
				3. & $\bigcirc$ & $\bigcirc$ & $\bigcirc$ & $\bigcirc$				
			\end{tabular}
		\end{center}
	}


	\item{
		\textbf{Haben Sie schon einmal eine größere Textmenge annotiert?}\\*
		\begin{center}
			\begin{tabular}{c | c}
					ja & nein \\
					$\bigcirc$ & $\bigcirc$ 
			\end{tabular}
		\end{center}
	}
	
	\item{
		\textbf{Wenn ja, welche Textsorten haben Sie bearbeitet?}\\*

			\begin{tabular}{c l}
				$\Box$ & Zeitungsartikel \\
				$\Box$ & Gesprochene Daten \\
				$\Box$ & Historische Texte \\
				$\Box$ & Lernertexte \\
				$\Box$ & Fragebögen \\
				$\Box$ & Dialektdaten \\
				$\Box$ & Parallelkorpora \\
				$\Box$ & Webkorpora \\
				$\Box$ & \underline{\ \ \ \ \ \ \ \ \ \ \ \ \ \ \ }
			\end{tabular}

	}
	
	\item{
		\textbf{Was für Annotationen haben Sie \emph{manuell} vorgenommen?}\\*

			\begin{tabular}{c l}
				$\Box$ & Wortart (POS-Tagging) \\
				$\Box$ & Lemmatisierung \\
				$\Box$ & Morphologische Annotation \\
				$\Box$ & Phonologische/prosodische Annotation \\
				$\Box$ & Syntax (Parsing) \\
				$\Box$ & Informationsstruktur \\
				$\Box$ & Semantische Annotation (Sense Tagging) \\
				$\Box$ & Koreferenz \\
				$\Box$ & \underline{\ \ \ \ \ \ \ \ \ \ \ \ \ \ \ }
			\end{tabular}
	}
	
	\item{
		\textbf{Was für Annotationen haben Sie \emph{automatisiert} vorgenommen?}\\*

			\begin{tabular}{c l}
				$\Box$ & Wortart (POS-Tagging) \\
				$\Box$ & Lemmatisierung \\
				$\Box$ & Morphologische Annotation \\
				$\Box$ & Phonologische/prosodische Annotation \\
				$\Box$ & Syntax (Parsing) \\
				$\Box$ & Informationsstruktur \\
				$\Box$ & Semantische Annotation (Sense Tagging) \\
				$\Box$ & Koreferenz \\
				$\Box$ & \underline{\ \ \ \ \ \ \ \ \ \ \ \ \ \ \ }
			\end{tabular}
	}
	
	\item{
		\textbf{Sind diese Daten in einem Korpus-Interface verfügbar? Es kann sich auch um ein nicht öffentliches Korpus handeln.}\\*
		\begin{center}
			\begin{tabular}{c | c}
				ja & nein\\
				$\bigcirc$ & $\bigcirc$
			\end{tabular}
		\end{center}
	}
	
	\item{
		\textbf{Mit welchen der folgenden Programmiersprachen oder Datenbanksystemen haben Sie tiefergehende Erfahrungen?}\\*

			\begin{tabular}{c l}
				$\Box$ & SQL / MS Access \\
				$\Box$ & C(++) oder Java \\
				$\Box$ & Perl / Python / Ruby \\
				$\Box$ & PHP / ASP \\
				$\Box$ & Prolog oder Lisp
			\end{tabular}
	}
	
	\item{
		\textbf{Geben Sie wenn möglich an, wie viele Teilstrings die folgenden \emph{regulären Ausdrücke} im Wort \emph{Rokkokokomode} finden.}\\*

			\begin{tabular}{c l}
				\underline{\ \ \ } & \verb! ok ! \\
				\underline{\ \ \ } & \verb! .{12} ! \\
				\underline{\ \ \ } & \verb! ^.{12}$ ! \\
				\underline{\ \ \ } & \verb! ok+ !
			\end{tabular}
	}
	
	\item{
		\textbf{Aus wievielen Wortformen besteht der folgende Text?}
		\begin{verbatim}
		Der Meister sagt: "Mit einem horizontal arbeitenden ein-ebenen 
		Personenfahrstuhl kann man z.B. nicht auf- und abfahren." 
		(s. Abb. 13).
		\end{verbatim}

			\begin{tabular}{l}
				\underline{\ \ \ \ \ }
			\end{tabular}
	}
	
	\item{
		\textbf{Ordnen Sie die folgenden Informationsquellen danach, wie wichtig diese für Ihren Einstieg in die Korpusauswertung waren.}\\*
		
		\begin{tabular}{c l}
			\underline{\ \ \ } & Einführungskurs / Workshop \\
			\underline{\ \ \ } & Online-Tutorial (zielorientierte Hilfe) \\
			\underline{\ \ \ } & Online-Hilfe (funktionsorientiere Hilfe) \\
			\underline{\ \ \ } & Dokumentation \\
			\underline{\ \ \ } & Persönliche Beratung
		\end{tabular}
	}
	
	\item{
		\textbf{Wie würden Sie die Benutzung eines neuen Systems am liebsten erlernen?}\\*
		
		\begin{center}
			\begin{tabular}{l r c c c c c l}
				& (eher nicht) & 1 & 2 & 3 & 4 & 5 & (sehr gern) \\
				Kurs & & $\bigcirc$ & $\bigcirc$ & $\bigcirc$ & $\bigcirc$ & $\bigcirc$ \\ \hline
				Online- Tutorials oder Hilfe & & $\bigcirc$ & $\bigcirc$ & $\bigcirc$ & $\bigcirc$ & $\bigcirc$ \\ \hline
				Dokumentation & & $\bigcirc$ & $\bigcirc$ & $\bigcirc$ & $\bigcirc$ & $\bigcirc$	\\ \hline	
			\end{tabular}
		\end{center}
	}
	
	\item{
		\textbf{Angenommen Sie haben einen Wunsch für Ihr "perfektes" Korpussystem frei. Was würden Sie sich wünschen?}\\*
		
			\begin{tabular}{l}
				\underline{\ \ \ \ \ \ \ \ \ \ \ \ \ \ \ \ \ \ \ \ \ \ \ \ \ \ \ \ \ \ \ \ \ \ \ \ \ \ \ \ \ \ \ \ \ } \\
				\underline{\ \ \ \ \ \ \ \ \ \ \ \ \ \ \ \ \ \ \ \ \ \ \ \ \ \ \ \ \ \ \ \ \ \ \ \ \ \ \ \ \ \ \ \ \ } \\
				\underline{\ \ \ \ \ \ \ \ \ \ \ \ \ \ \ \ \ \ \ \ \ \ \ \ \ \ \ \ \ \ \ \ \ \ \ \ \ \ \ \ \ \ \ \ \ }
			\end{tabular}
	}


\end{enumerate}
